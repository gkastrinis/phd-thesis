\label{chapter:intro}

\epigraph{Sometimes a scream is better than a thesis.}{\textit{Manfred Eigen}}

Static program analysis is the cornerstone of several modern programming facilities and tools for program development and aided program understanding. Nowadays, it is an umbrella term for many different methodologies (...) all with the ultimate goal of inferring a program's properties, without the need of an actual execution. It is routinely employed in many different contexts: compilers, bug detectors, verifiers, security analyzers, IDEs, and a myriad other tools. 

The main intention of any \emph{static} program analysis algorithm is to reason about the set of all feasible behaviors (under some abstraction of behaviors) that a given program might exhibit under all possible executions. For example, could this method throw a runtime exception, or is that type cast possible to fail under some program input, etc.

Pointer analysis (also known as \emph{points-to analysis}) is a fundamental subdomain of static program analysis that consists of computing some \emph{abstract memory model} for a given program. The essence of such an analysis is to compute a set of possible objects that a program variable or expression may point to during program execution. A straightforward endeavor at first, it quickly gets too complicated in practice due to all of the intricate details one has to take into account and the multitude of different features that mutually depend on each other. Although a challenging task, if implemented correctly and not naively it can bear many significant benefits to client analyses that consume its results to reason about specialized behaviors such as security vulnerabilities or potential optimization opportunities.

A closely related analysis, sometimes wrongfully confused with pointer analysis, is \emph{alias analysis} in which one computes sets of program expressions that may alias (i.e., point to common objects) with each other. Pointer analysis could --although it is not the only possible alternative-- be used to implement an alias analysis algorithm, and vice versa.

At the same time, programming languages are ever evolving, ever becoming more high-level and more complex. Many abstraction levels are added throughout the years with the aim of making the very task of programming easier for developers allowing them to express more with less effort (e.g., in terms of lines of code). Frequently, new features come with complicated semantics regarding their possible implementations and usually they interact in intricate ways with pre-existing ones.

Additionally, modern software paradigms have evolved as well. Complex design patterns have become the norm for experienced developers, immense libraries and frameworks are accepted as a prerequisite for any non-trivial software, and over-involved build tools often make even the task of understanding all of the program's dependencies a challenge.

It comes as no surprise that any kind of static analysis has struggled to keep up with this ever-increasing complexity both in programming languages and software. Even the seemingly simple task of computing a program's call-graph (i.e., which program functions can call which others), if one tries to have an approach as general as possible, requires sophisticated analysis in order to achieve the desirable precision. Thus, the main emphasis of pointer analysis algorithms is on combining fairly precise modeling of pointer behavior and memory abstractions with scalability.

\paragraph*{Thesis.}
\begin{displayquote}
It is possible to implement \emph{highly sophisticated} and \emph{precise} static pointer analysis algorithms without forgoing \emph{scalability}. Furthermore, precision and the accompanied \emph{confidence in results} is a spectrum and can be tweaked differently for different parts of the program.
\end{displayquote}

We provide a number of techniques for implementing scalable static pointer and alias analyses in the setting of Java programs by configuring the analysis strategy differently for different code parts. Additionally, we present a couple of defensive algorithms for reporting highly-confident results even in the presence of hostile and/or unknown program points.


\section{Context-Sensitivity}

Throughout the years, pointer analysis has evolved and has been the focus of intense research. It is widely accepted to be among the most standardized and well-understood inter-procedural (i.e., reasoning about a property taking into account the flow of code across different program functions) analyses.

A widely used concept that emerged as a powerful tool for tuning precision while still achieving scalable analyses, is that of \emph{context-sensitivity}. It consists of qualifying interesting components of an analysis, such as program expressions, object abstractions or method invocations, with additional \emph{context} information. The core idea being that the analysis will collapse information (e.g., ``what objects this method argument may point to'') for components that result in the same context, while keep separate information for different contexts. In essence, qualifying components with additional context is as if each such component is replaced with multiple versions (one for each different associated context value) and the analysis can reason individually for each version.

Two main flavors of context-sensitivity have been explored in past literature: \begin{inparaenum}[(1)]
\item \emph{call-site sensitivity} (also known as \emph{kCFA}) in which call-sites are used to qualify variables and other analysis components, essentially re-analyzing a method for different call-sites that target that method, and
\item \emph{object-sensitivity} in which receiver objects of a call are used instead, in a similar manner.
\end{inparaenum}
Another kind of context-sensitivity, known as \emph{type-sensitivity}, has also been explored as an approximation of object-sensitivity with the aim of preserving high precision at substantially reduced cost. In type-sensitivity, upper bounds on the dynamic types of the receiver objects are employed as context elements. 

A context-sensitive analysis has a second axis of parameterization besides context flavor --that of (max) context depth. Consequently, a common way to describe an analysis is using the following naming scheme: $X$-$FLAVOR$-sensitive+$Y$-heap, e.g., as in 3-call-site-sensitive+2-heap. Here $FLAVOR$ denotes the kind of context information being employed, and, $X$ and $Y$ denote the context depth limits being used at invocation sites and at object allocations respectively. In the previous example the analysis is keeping track of the 3 most recent call-stack frames that led to the current method call, in order to annotate local variables. Similarly, the analysis is using the 2 most recent call-stack frames that led to the allocation site of an object to annotate the newly allocated object.


\section{Design Choices}

There are a few important design choices that can affect drastically the properties a static program analysis algorithm will enjoy and the reasoning that each one will need to implement.

\subsection{Whole-Program vs. Partial Analyses}

Different static program analyses may define differently the parts of the program that are of interest. A \emph{whole}-program analysis examines every part of the program, including any external dependencies that the code may have, and reasons about the effects each part has on the rest of the program. In the setting of Java for example, a whole-program analysis such as an analysis regarding \emph{thrown exceptions} not only reasons about the application code but additionally about any third-party library used by the program (e.g., from external Java Archives --JARs) and also about code run by the Java Runtime Environment --i.e, library code provided by the language itself.

On the other hand, a \emph{partial}-analysis only focuses on specific parts of the program and ignores the effects of the rest. Multiple analyses commonly found in a classical compiler, such as \emph{type-checking} or the computation of \emph{live-ranges} for program expressions are well-known examples of partial static program analyses.

A partial-analysis algorithm usually may afford to implement more complex, more expensive reasoning than a whole-program one, since it only focuses on a very localized part of the program. On the contrary, a whole-program analysis has to constantly balance the complexity of its logic,  any potential precision gains but also any scalability penalties. The rest of this dissertation will only focus on a few interesting whole-program analyses.

\subsection{Flow-Sensitivity} \label{flowSensitivity}

Although, counter-intuitive at first, it is not unusual for a static program analysis to be flow-\emph{insensitive}. A flow-sensitive analysis examines a method's instructions while taking the order they appear in the source code into account. On the contrary, a flow-insensitive analysis examines a method's instructions as if they were in a set, without any particular order (i.e., any instruction could happen before any other). The latter approach leads to analyses that overapproximate the semantics of the actual code --thus suffering in precision-- but it is a common tradeoff when aiming to improve the performance of an analysis.

The penalties on performance for a flow-sensitive analysis mainly stem from the need to keep track of what holds at \emph{every single} program point. Potentially, this could mean that information that remains unchanged will be duplicated on a multitude of instructions. On the contrary, a flow-insensitive analysis will collapse information along all instructions of a method.

{
\setlength\intextsep{-10pt}
\begin{wrapfigure}{ht}{.12\textwidth}
\centering
\begin{javacodeLines}
x = 1;
y = x;
x = 2;
\end{javacodeLines}
\end{wrapfigure}

For the example on the side, a flow-sensitive analysis reasoning about the values of primitive expressions might report that:
after line 1: ``$x$ has the value 1'',
after line 2: ``$x$ has the value 1'' and ``$y$ has the value 1'', and
after line 3: ``$x$ has the value 2'' and ``$y$ has the value 1''.

A flow-insensitive analysis might instead report that:
``$x$ has either the value 1 or 2'' and ``$y$ has either the value 1 or 2'',
because instructions are examined as if happening in any order.
}

\subsection{Static Single Assignment Form}

In compiler design, \emph{static single assignment form} (also known as SSA) is a kind of code transformation in which every local variable is assigned only once. Existing local variables in the original source code are split into \emph{versions} (e.g., variable $x$ might be split into $x_1$ and $x_2$) with each version being assigned only once. At program points where the value of the original variable is read and there are multiple valid versions, as in the point where the branches of an \emph{if-else} statement merge, a \emph{phi-node} statement is used. This special statement bears the semantics of somehow ``choosing'' a specific variable version to read.

\begin{figure}[h]
\begin{subfigure}{.45\textwidth}
\begin{javacodeLines}
if (...) x = 10;
else x = 20;
y = x;
\end{javacodeLines}
\caption{Original source code}
\end{subfigure}%
    \hfill
\begin{subfigure}{.45\textwidth}
\begin{javacodeLines}
if (...) x_1 = 10;
else x_2 = 20;
y = phi(x_1, x_2);
\end{javacodeLines}
\caption{The equivalent SSA form}
\end{subfigure}
\end{figure}

In the context of static program analysis, SSA is often used to approximate the benefits of a flow-sensitive analysis, particularly pertaining to the handling of local variables. This is not the case for other, more complicated language features such as heap accesses and method invocations, but SSA provides an easy way to pick the low-hanging fruit.

\setlength\intextsep{10pt}
\begin{wrapfigure}{ht}{.12\textwidth}
    \centering
\begin{javacodeLines}
x_1 = 1;
y = x_1;
x_2 = 2;
\end{javacodeLines}
\end{wrapfigure}

The flow-\emph{insensitive} analysis of \ref{flowSensitivity} will report quite different results when analyzing the SSA form equivalent of the example code, that is given on the side: ``$x_1$ has the value 1'', ``$y$ has the value 1'', and ``$x_2$ has the value 2''. The analysis is still examining instructions without taking order into account, and is unable to answer questions such as ``what is the value of variable $x$ at the end of the method'', but nevertheless it has managed to reclaim some of the precision that was previously lost --i.e., regarding the value of variable $y$.

\subsection{May vs. Must Analyses}

Given an abstraction of behaviors (e.g., thrown exceptions) one can define two interesting sets regarding the potential behaviors that a given program will exhibit. Set $Any(P)$ is defined as containing all possible behaviors that a program $P$ \emph{will} exhibit in \emph{some} program execution (e.g., method $m1$ will throw exception $e1$ in one execution and exception $e2$ in another). Respectively, set $All(P)$ is defined as containing behaviors that will appear in \emph{every} program execution (e.g., method $m2$ will always throw exception $e3$ during any execution). I.e.,
\[
Any(P) := \bigcup_{e \in Executions} Behaviors(P, e)
\quad \textrm{and} \quad
All(P) := \bigcap_{e \in Executions} Behaviors(P, e)
\]

Both sets are a mathematical ideal, an answer only an oracle could provide. But, for any realistic analysis such an endeavor is an undecidable problem. Thus, a practical static program analysis will aim to compute an approximation of one of the two sets. Under that definition, analyses are split mainly into two groups depending the kind of approximation they try to achieve. On one hand, a \emph{may}-analysis is one that aims to \emph{over}-approximate $Any(P)$. On the other hand, a \emph{must}-analysis is one that aims to \emph{under}-approximate $All(P)$. I.e.,
\[
May(P) \supseteq Any(P) \quad \textrm{and} \quad Must(P) \subseteq All(P)
\]

\section{Soundness \& Completeness}

The term \emph{soundness} (and its converse \emph{completeness}) originate from formal mathematical logic where it is used in order to evaluate a \emph{proof} system under a given \emph{model}. The model is some kind of mathematical structure, such as sets over some domain of interest and the proof system is a set of rules with which \emph{properties} regarding the model are proven. A system is sound if and only if statements it can prove are indeed true in the model (concisely given as ``claim implies truth''). A system is complete if and only if what is true in the model can also be proven by the system (concisely given as ``truth implies claim'').

In the context of static program analysis the most relevant and widely used term is that of soundness. In this setting, the analysis is making claims regarding potential program behaviors under any program execution and the validity of those claims constitutes whether the analysis is sound or not. More specifically, a may-analysis is sound whenever what it claims is actually true --i.e., the computed behaviors are an overapproximation of $Any(P)$. Respectively, a must-analysis is sound whenever the computed behaviors are an underapproximation of $All(P)$. A trivially sound may-analysis could simply infer top ($\top$), i.e., every possible behavior. A trivially sound must-analysis could simply infer the empty set ($\emptyset$), i.e., no behavior at all.

Contrary to the prevalent use of soundness for evaluating static program analysis algorithms, completeness is scarcely referenced --if ever. One can easily find ``proofs of soundness'' on many publications but not the analogous ``proofs of completeness''. This is mainly because of the way analyses (may- or must-) are defined as aiming to compute an \emph{approximation} of potential behaviors. For example, what would the meaning of a complete may-analysis be? Such an analysis has to abide by the definition of ``truth implies claim''. In this case ``truth'' is any overapproximation of $Any(P)$. Consequencently, a ``complete'' may-analysis would need to compute all possible overapproximations of $Any(P)$, and thus, the term is less relevant in the domain of static program analysis.

Additionally to soundness, there are two closely related terms characterizing the validity of each claim made by the analysis. If the analysis incorrectly claims that some behavior is among the potential program behaviors, then this constitutes a \emph{false-positive}. E.g., the analysis claims that method $m1$ might throw an exception $e1$, but there is no program execution under which this will actually occur. Similarly, if an analysis claims that some behavior will never happen but in reality there exists a program execution where such a behavior takes place, then this constitutes a \emph{false-negative}. E.g., the analysis claims that method $m2$ will never throw an exception, but it actually does.

It is noteworthy that every static program analysis is also making negative claims, in addition to positive ones, even if only implicitly. This is due to an analysis not making some claim actually implicitly claiming its negation. For example, if a may-analysis for thrown exceptions reports that method $m1$ might throw either exception $e1$ or exception $e2$, then it also implicitly reports that the same method will never throw any other exception --given that the analysis aims to compute an overapproximation of possible behaviors.

\subsection{Precision \& Recall}

Both a may- and a must-analysis operate under the premise of computing an approximation of reality and thus there will always be claims that are either extraneous or missing. Of course, any analysis should do its best to get as close to the truth even if the actual set of behaviors will always be out of grasp. This calls for some \emph{quantitative} way to measure an analysis's quality and two such metrics have been proposed. \emph{Precision} indicates how many of the analysis claims are actually part of the truth, whereas \emph{recall} indicates how many of the actual true claims are also reported by the analysis. For a formal definition supposing that:
\begin{itemize}
    \item $X$ is the number of true (actually happening) interesting behaviors
    \item $T \leq X$ is the number of correct claims made by the analysis
    \item $F$ is the number of incorrect claims made by the analysis
\end{itemize}
then
\[
Precision(P) = \frac{T}{T + F}
\quad \textrm{and} \quad
Recall(P) = \frac{T}{X}
\]

A value of 1 is a perfect score, and a value of 0 is the worst one. A sound may-analysis will have perfect recall, since it at least claims any behavior that is actually true. A sound must-analysis will have perfect precision, since it makes no incorrect claims (it does not report any behavior that can not actually occur).

Highly useful as they are, both measures have two major shortcomings. Firstly, these are \emph{empirical} measures in the sense that they measure an analysis's performance in regards to a specific given program. They bear no information about the analysis behavior at a theoretical level and cannot be generalized to other programs. An analysis could be perfect for one program and terrible for another. This could be tackled to some extend with the use of well established benchmarking suites during the testing phase, that aim to cover common and interesting code patterns and program behaviors.

Secondly, what might be the most challenging issue, it is rarely the case that there is actual knowledge of the ground-truth. Somewhat a chicken-and-egg problem, having an automatic way to retrieve the ground-truth for arbitrary programs is both what a static analysis aims to achieve at its core and also what is needed to evaluate its claims. Usually, one has to resort to observing a limited amount of actual executions for a given program and, making the assumption that the observations are a good enough representative, interpolate to all possible program executions.

\subsection{Soundiness}

Although soundness seems like an essential property for any static program analysis algorithm to have, and is quite prevalent in academic literature, Livshits et al [...] make a strong claim in that there is no practical sound \emph{whole-program} may-analysis. Most of the time, this is a conscious compromise on how to handle certain language features and not due to lack of understanding. If one would attempt to soundly model such features in the context of a may-analysis, i.e., overapproximate their effects, this would most probably result in an analysis that is so unscalable or imprecise that is practically \emph{useless}.

At the same time, many academic publications make claims of soundness and may even provide some kind of ``proof of soundness'' but this is mostly in regards to a \emph{subset} of language features --the analysis might be totally unsound in handling all the rest. Thus, a need arises for a way to differentiate between an analysis that tries its best to be sound and only gives up in a well-defined language subset, and one that is simply unsound.

The term \emph{soundiness}, specific to the context of static program analysis, was coined by [..] for such a purpose. A \emph{soundy} analysis handles most classical, core language features in a sound manner (i.e., overapproximates) and may only fail to do so (i.e., underapproximates) in a small well-accepted subset of highly \emph{dynamic features}, specific to each language. Such features include uses of \emph{reflection} or \emph{native code} in Java, \emph{eval} in Javascript, and \emph{pointer arithmetic} in C/C++.


\section{Scientific Contributions}

In this section, we will briefly explain the main scientific contributions of this dissertation.

Ever since the introduction of object-sensitivity by Milanova et al. [..], there has been increasing evidence that it is the superior context choice for programs expressed in object-oriented languages, yielding a high precision to cost ratio. Such has been its success that in practice it has almost superseded the use of more traditional call-site sensitive analyses in regards to object-oriented languages. Nevertheless, a call-site sensitive analysis is not always inferior as there are language features and code patterns that may favor this kind of context abstraction --at least in certain program points.

Subsequently, a naive approach would be combining both context flavors with the goal of increasing the resulting analysis precision. Truly, such a combination would bear precision benefits but in most cases it is accompanied by an infeasibly high and disproportionate cost.

Our first scientific contribution...


\section{Outline}

The rest of this dissertation is organized as follows:
\begin{itemize}%[--]
\item Chapter~\ref{chapter:hybrid} bla blaasda

    This chapter presents research previously published in...
  %\emph{\citetitle{foo}} \cite{foo}.

\item Chapter~\ref{chapter:introspective} bla bla

    This chapter presents research previously published in...

\item Chapter~\ref{chapter:must-logic} bla bla

    This chapter presents research previously published in...

\item Chapter~\ref{chapter:must-data} bla bla

    This chapter presents research previously published in...

\item Chapter~\ref{chapter:defensive} bla bla

    This chapter presents research previously published in...
    
\item Chapter~\ref{chapter:panda} bla bla

\item Chapter~\ref{chapter:related} first discusses related work that
  is specific to the previous chapters, and then expands to various
  other interesting subjects in the broader realm of static analysis.
%   Some parts of this chapter are based on the aforementioned papers
%   \cite{structsens,reflection,jphantom}, and some on the survey
%   \emph{\citetitle{survey}} \cite{survey}.

\item Chapter~\ref{chapter:conclusions} concludes this dissertation by assessing our initial thesis and discussing future work.
\end{itemize}