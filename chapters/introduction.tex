\label{chapter:intro}

\epigraph{Sometimes a scream is better than a thesis.}{\textit{Manfred Eigen}}

Static program analysis is the cornerstone of several modern programming facilities and tools for program development and aided program understanding. Nowadays, it is an umbrella term for many different methodologies (...) all with the ultimate goal of inferring a program's properties, without actually running it. It is routinely employed in many different contexts: compilers, bug detectors, verifiers, security analyzers, IDEs, and a myriad other tools.

At the same time, programming languages are ever evolving, ever becoming more high-level and more complex. Many abstraction levels are added throughout the years with the aim of making the very task of programming easier for developers allowing them to express more with less effort (e.g., in terms of lines of code). Frequently, new features come with complicated semantics regarding their possible implementations and usually they interact in intricate ways with pre-existing ones.

Additionally, modern software has evolved as well. Complex design patterns have become the norm for experienced developers, immense libraries and frameworks are accepted as a prerequisite for any sophisticated software, and over-involved build tools often make even the task of understanding all of the program's dependencies a challenge.

It comes as no surprise that any kind of static analysis has struggled to keep up with this ever-increasing complexity both in programming languages and software. Even the seemingly simple task of computing a program's call-graph (i.e., which program functions can call which others) requires sophisticated analysis in order to achieve precision, if one tries to have an approach as general as possible.

Pointer analysis (also known as \emph{Points-To analysis}) is a fundamental subcategory of static program analysis that consists of computing some \emph{abstract memory model} for a given program. The essence of such an analysis is to compute a set of possible objects that a program variable or expression may point to during program execution. Although, seemingly a simple task it is



\paragraph*{Thesis.}
\begin{displayquote}
what
\end{displayquote}