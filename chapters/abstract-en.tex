Static program analysis aims to automatically reason about certain properties a given program might exhibit under all possible executions without actually observing such executions. Static \emph{pointer} analysis is a major subcategory that focuses on the objects that program expressions might point to during program executions. The evolution of programming languages has led to the addition of many abstraction layers that, as a result, have made any automatic reasoning about a program a challenging task at best or an infeasible one at worst. Thus, any practical static pointer analysis algorithm has to compromise and aim to approximate results in some way---either computing more or less than what is actually true.

This dissertation shows how we can obtain \emph{precise} yet also \emph{scalable} static pointer analysis algorithms by carefully differentiating policies for different parts of the program. Furthermore, since a static pointer analysis algorithm with global soundness guarantees and meaningful results throughout is not realistic, we show that it is possible to design analyses that offer \emph{strong guarantees} on the soundness of the results for specific parts of the program.

Pointer analyses in the past introduced the concept of \emph{context-sensitivity} in order to tackle the ever growing problem of imprecision versus scalability. Context is used to annotate analysis components so that the analysis can be more precise without at the same time sacrificing scalability. We show beneficial ways to combine different context flavors for different parts of the program without paying the cost that a naive combination would incur.

Another attempt at producing precise yet scalable analyses leads us to an introspective analysis. We employ a common adaptive pattern in which a cheap imprecise analysis is run first so various metrics can be gathered, and then a more precise (and costly) analysis can be used only in parts of the program---under the assumption that more precise handling of the rest would only incur performance penalties.

Subsequently, we shift our attention to an analysis that \emph{under}-approximates results (instead of the norm of \emph{over}-approximating) so that it might report less but can guarantee those properties to always hold. We build upon observations on the properties that such analyses have in order to apply a specialized data structure that speeds up our algorithm by nearly two orders of magnitude.

Finally, in our last contribution, we revisit an analysis formulation that over-approximates results to create an analysis algorithm that is truly sound but at the same time highly efficient. Our analysis is conservative, guaranteeing soundness even in the presence of arbitrary unknown code, but avoids wasting any work on computations that will later be invalidated due to soundness concerns.