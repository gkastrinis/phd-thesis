\chapter{Related Work}\label{chapter:related}

\section{Hybrid}

We have discussed directly related work throughout the paper. Here we
selectively mention a few techniques that, although not directly
related to ours, offer alternative approaches to sweet spots in the
precision/performance tradeoff.

Special-purpose combinations of context-sensitivity have been used in
the past, but have required manual identification of classes to be
treated separately (e.g., Java collection classes, or library factory
methods). An excellent representative is the TAJ work for taint
analysis of Java web applications
\cite{pldi:2009:Tripp}. In contrast, we have sought to
map the space and identify interesting hybrids for general application
of context-sensitivity, over the entire program.

The analyses we examined are context-sensitive but flow-insensitive.
We can achieve several of the benefits of flow-sensitivity by applying
the analysis on the static single assignment (SSA) intermediate form
of the program. This is easy to do with a mere flag setting on the
\doop{} framework. However, the impact of the SSA transformation
on the input is minimal. The default intermediate language used as
input in \doop{} (the Jimple representation of the Soot
framework \cite{cascon:1999:Vall,cc:2000:Vall}) is already close to
SSA form, although it does not guarantee that every variable is
strictly single-assignment without requesting it explicitly.  Recent
work by Lhot\'{a}k and Chung \cite{popl:2011:Lhotak} has shown that
much of the benefit of flow-sensitivity derives from the ability to do
strong updates of the points-to information. Lhot\'{a}k and Chung then
exploited this insight to derive analyses with similar benefit to a
full flow-sensitive analysis at lower cost.

A demand-driven evaluation strategy reduces the cost of an analysis by
computing only those results that are necessary for a client program
analysis~\cite{oopsla:2005:Sridharan,pldi:2006:Sridharan,popl:2008:Zheng,pldi:2001:Heintze}. This is a useful
approach for client analyses that focus on specific locations in a
program, but if the client needs results from the entire program, then
demand-driven analysis is typically slower than an exhaustive
analysis.

Reps~\cite{cc:1994:Reps} showed how to use the standard magic-sets
optimization to automatically derive a demand-driven analysis
from an exhaustive analysis (like ours). This optimization
combines the benefits of top-down and bottom-up evaluation of
logic programs by adding side-conditions to rules that limit
the computation to just the required data.

An interesting recent approach to demand-driven analyses was
introduced by Liang and Naik \cite{pldi:2011:Liang}.
Their ``pruning'' approach consists of first computing a coarse
over-approximation of the points-to information, while keeping the
provenance of this derivation, i.e., recording which input facts have
affected each part of the output. The input program is then pruned so
that parts that did not affect the interesting points of the output
are eliminated. Then a precise analysis is run, in order to establish
the desired property.


\section{Introspective}

The effort to tune the context-sensitivity of an analysis is pervasive
in the literature. Nevertheless, most approaches fundamentally differ
from ours, either by trying to vary context-sensitivity based on
syntactic properties or by trying to focus on only a part of the program
that matters for answering a given query.
In contrast, we attack the context-sensitive scalability problem 
head-on, in the all-points points-to analysis setting, with context
used all over the program and library.

Typical scalable points-to analysis frameworks such as
Wala~\cite{www:wala} and Doop~\cite{oopsla:2009:Bravenboer} employ a multitude of
low-level heuristics for tuning the precision and scalability of an
analysis. These include using extra context for collection classes,
using a heap context for arrays in an analysis without a
context-sensitive heap, allocating strings or exceptions
context-insensitively, treating library factory methods with deeper
context, etc. Such heuristics are typically user-selected and
prominent in the documentation of the respective frameworks, and have
also appeared in the literature (e.g.,
\cite{pldi:2009:Tripp,cc:2013:Kastrinis}).  However, all
such approaches are mere hard-wired heuristics and do not address the
major scalability problem that our approach aims to solve. The
scalability issues identified in earlier literature and discussed
throughout this paper are present after all such heuristics have been
employed.
% These
%heuristics are already implemented in the baseline framework that we
%use for our analysis. 

A more general approach is \emph{hybrid} context-sensitivity, which
consists of treating virtual and static method calls differently
\cite{pldi:2013:Kastrinis}. Such a hybrid analysis attempts to emulate
call-site sensitivity for static method calls and object-sensitivity
for dynamic calls. The approach becomes interesting when context is
deep (e.g., how are context elements merged when a dynamic call is
made inside a static call?). Nevertheless, the hybrid
context-sensitivity approach does not change the essence of the
problem we are trying to solve. For hard-to-analyze applications,
hybrid context-sensitive algorithms are equally unscalable as their
component algorithms. For the purposes of our experimental study,
which only tests the scalability of heavyweight benchmarks, hybrid
context-sensitivity is virtually indistinguishable from object-
sensitivity.

More interesting applications of selective context-sensitivity have
been explored in the context of \emph{demand-driven} pointer analysis.
A demand-driven evaluation strategy reduces the cost of an analysis by
computing only those results that are necessary for a client program
analysis~\cite{oopsla:2005:Sridharan,pldi:2006:Sridharan,popl:2008:Zheng,pldi:2001:Heintze}. This is a useful
approach for client analyses that focus on specific locations in a
program, but if the client needs results from the entire program, then
demand-driven analysis is typically slower than an exhaustive
analysis.

In the demand-driven space, refinement-based analyses have been used
primarily in the work of Sridharan and Bod\'{\i}k~\cite{pldi:2006:Sridharan} and
of Liang and Naik~\cite{pldi:2011:Liang}.  Sridharan
and Bod\'{\i}k~\cite{pldi:2006:Sridharan} introduce refinement-based analysis as a
way to adaptively increase the precision characteristics of an
existing analysis algorithm when a client analysis is not satisfied
with the result. The approach allows turning on field-sensitivity, as
well as higher call-site sensitivity for an analysis algorithm. Yet,
unlike ours, it is not a general approach that can apply to any kind
of context and a large number of different algorithms.  Liang and
Naik's ``pruning'' approach \cite{pldi:2011:Liang}
consists of first computing a coarse over-approximation of the
points-to information, while keeping the provenance of this
derivation, i.e., recording which input facts have affected each part
of the output. The input program is then pruned so that parts that did
not affect the interesting points of the output are eliminated. Then a
highly context-sensitive precise analysis is run, in order to
establish the desired property. This approach is similar to
introspective context-sensitivity in that the analysis is run twice
and a separate query over the first-run result determines the second
run's characteristics.  Nevertheless, our approach requires no
provenance computation (which is unlikely to scale for an all-points
analysis) and works even when we want answers for the entire
program---i.e., when pruning is not possible.

Both of the above demand-driven approaches can be viewed as
complements of our introspective context-sensitivity. In the
demand-driven world, it is possible to estimate the \emph{benefit}
that a more precise analysis may yield: either the client is happy
with the current level of precision (which implies there is no further
benefit to be obtained) or it is not, in which case more precision
should be added. In our all-points pointer analysis problem we have no
such information. This motivates our \emph{cost}-based heuristics,
which attempt to estimate ``what can go wrong'' when more precision
gets added, as opposed to ``what can be gained'', as in demand-driven
techniques.


\section{Must-Analysis: Logic}

There are several approaches in the literature that present
must-analyses in the pointer analysis setting or employ them in a
may-analysis. Our approach is a must-alias analysis applied to Java
bytecode, but conceptually it is distinguished by its minimizing the
distance between the implementation and the declarative specification.
%% and by its leveraging of a highly-efficient data structure for sets of
%% alias classes.

%% Ma et al.~\cite{isola:2008:Ma} present an algorithm for
%% null-pointer dereference detection using a
%% context-insensitive may-alias and a must-alias analysis; the latter is
%% used to increase the precision of the former, by enabling strong
%% updates when possible.

Nikoli\'{c} and Spoto~\cite{ictac:2012:Nikolic} present a
must-alias analysis that tracks aliases between program expressions
and local variables (or stack locations, since they analyze Java
bytecode%% , which is a stack-based representation
).  The analysis
%itself does not expose any clear configuration points but it 
is related to ours both because of its application to Java bytecode
and because it is constraint-based: the analysis is a generator of
constraints, which are subsequently solved to produce the analysis
results.
% Abstractly,
%this is a relative of our Datalog-based approach, but it is unclear
%how the two may compare in terms of engineering tradeoffs.

Hind et al. \cite{article:1999:Hind} present a collection of pointer
analysis algorithms. Among them, the most relevant to this work is a
flow-sensitive interprocedural pointer alias analysis. The authors
optimistically produce \emph{must} information for pointers to single
non-summary objects.

Emami et al.~\cite{pldi:1994:Emami} present an approach that
simultaneously calculates both must- and may-point-to information for
a C analysis. Their empirical results ``show the existence of a
substantial number of definite points-to relationships, which forms
very valuable information''---much in line with our own
experience.

Must- information is often computed in conjunction with a
client analysis. One of the best examples is the typestate
verification of Fink et al.~\cite{issta:2006:Fink}, which demonstrates the value of
a must-analysis and the techniques that enable it.

The analysis of \cite{ecoop:2012:De} is essentially a
flow-sensitive may-point-to analysis that performs strong updates, as
it maps \emph{access paths} to \emph{heap objects} (abstracted by
their allocation sites). 
The approach uses a flow-insensitive
may-point-to analysis to bootstrap the main analysis. However, it
provides no \emph{definite} knowledge of any sort, since the aim is to
increase the precision of the may-analysis. For instance, even if an
access path points to a single heap object, according to the De and
D'Souza analysis, there is no \emph{must} point-to information
derived, since this object could be a summary object (i.e., one that
abstracts many objects allocated at the same allocation site). To
reason about such cases, other approaches, such as the more expensive
shape analysis algorithms \cite{article:2002:Sagiv},
additionally maintain summary information per heap object. In this
way, they allow must point-to edges to exist only if the target is
definitely not a summary node.

%An alternative approach to handle this shortcoming is to propagate
%% An approach for integrating \emph{must} point-to reasoning in an
%% analysis is to propagate such
%% %\emph{must}-point-to 
%% information only at instructions where we know that the given heap
%% allocation target still refers to the last object allocated at that
%% site \cite{Altucher:1995:EFM:199448.199466}. Thus, an execution path
%% that may create another object at the same site (such as when reaching
%% the end of the loop) would invalidate any previous must-point-to facts
%% (i.e., it will stop them from propagating any further).

Generally, must-analyses can vary greatly in sophistication and can be
employed in an array of different combinations with may-analyses.  The
analysis of Balakrishnan and Reps~\cite{sas:2006:Balakrishnan}, which
introduces the \emph{recency abstraction}, distinguishes between the
most recently allocated object at an allocation site (a concrete
object, allowing strong updates) and earlier-allocated objects
(represented as a summary node). The analysis additionally keeps
information on the size of the set of objects represented by a summary
node. At the extreme, one can find full-blown shape analysis
approaches, such as that of Sagiv et
al.~\cite{article:2002:Sagiv}, which explicitly maintains must-
and may- information simultaneously, by means of three-valued truth
values, in full detail up to predicate abstraction: a
relationship can definitely hold (``must''), definitely not hold
(``must not'', i.e., negation of ``may''), or possibly hold
(``may''). Summary and concrete nodes are again used to represent
knowledge, albeit in full detail, as captured by arbitrary predicates
whose value is maintained across program statements, at the cost of a
super-exponential worst-case complexity.

%% Jagannathan et al.~\cite{popl:1998:Jagannathan} present
%% an algorithm for must-alias analysis of functional languages. The
%% algorithm adapts must-alias insights to the setting of captured
%% variables.
%% %captured in closures.  
%% For instance, must-alias information for
%% non-summary objects permits strong updates, which the authors find to
%% improve analysis precision. We employ must-alias analysis results
%% quite similarly in applications of our model analysis.


%% Our optimized data structure is (partly) based on the observation that
%% must-alias sets are equivalence classes. This is not the first time
%% that a data structure that efficiently implements equivalence classes
%% has been used to speed up pointer analysis. Most notably, a
%% Steensgaard-style (or \emph{unification-based})
%% \cite{steensgard:1996:PointsTo} analysis computes may-point-to sets
%% that are equivalence classes. This means that points-to sets are
%% disjoint---if two points-to sets are found to possibly overlap, they
%% get unified. This loses precision (relative to a standard subset-based
%% points-to analysis) but enables the algorithm to use union-find trees
%% for a very efficient representation.

%% Another optimized data structure often used in pointer analysis is the
%% \emph{constraint graph}: a graph with nodes denoting pointer variables
%% and an edge between nodes \code{p} and \code{q} denoting flow (e.g., a
%% direct assignment) from variable \code{p} to variable \code{q}.  Online
%% cycle elimination by F\"{a}ndrich et al.
%% \cite{pldi/FahndrichFSA98} detects cycles in the
%% constraint graph and collapses all nodes in a cycle into a
%% representative node, since such nodes will have identical points-to
%% information. The technique of Nasre
%% \cite{ismm/Nasre12} extends such constraint graph
%% reasoning based on the observation that if two nodes have the same
%% dominator in the constraint graph, then they are clones: the values
%% flowing to them are (only) those of the dominator node. Several other
%% constraint graph optimizations are applied off-line (i.e., before the
%% points-to analysis runs).  Prime examples of such techniques are
%% Rountev and Chandra's \cite{rountevOffline} and Hardekopf and Lin's
%% \cite{hardekopfOffline}. (Hardekopf and Lin have also applied similar
%% ideas in a hybrid online/offline setting \cite{antgrasshopper}.)  Both
%% of these techniques perform an off-line detection of equivalent
%% points-to sets and use this knowledge to eliminate redundant work in
%% subsequent points-to computations. Our data structure can be seen as
%% somewhat analogous to constraint-graph techniques, in the sense that
%% we do not compute the flow of objects or the fully expanded set of all
%% possible alias pairs. Instead, we compute the ``wiring'' (i.e., the
%% alias relationships, locally, that the program induces) and keep the
%% alias information in condensed form, until it needs to be queried by a
%% client analysis.

%% %Hardekopf and
%% %Lin's approach is impressively general, computing hash codes that
%% %encode all the logical processing of a points-to set that is induced
%% %by the current program and, thus, detecting equivalent points-to sets
%% %even through complex program patterns.

%% % \citep{Heintze:2001:UAA:378795.378855} ; pre-transitive constraint-graph and reachability caching


\section{Must-Analysis: Datastructure}

There are several approaches in the literature that present
must-analyses in the pointer analysis setting or employ them in a
may-analysis. Additionally, there are several approaches that
integrate efficient data structures in the representation of
points-to information.


\paragraph{Data Structures and Heap Abstractions.}
Our optimized data structure is (partly) based on the observation that
must-alias sets are equivalence classes. This is not the first time
that a data structure that efficiently implements equivalence classes
has been used to speed up pointer analysis. Most notably, a
Steensgaard-style (or \emph{unification-based})
\cite{popl:1996:Steensgaard} analysis computes may-point-to sets
that are equivalence classes. This means that points-to sets are
disjoint---if two points-to sets are found to possibly overlap, they
get unified. This loses precision (relative to a standard subset-based
points-to analysis) but enables the algorithm to use union-find trees
for a very efficient representation.

Another optimized data structure often used in pointer analysis is the
\emph{constraint graph}: a graph with nodes denoting pointer variables
and an edge between nodes \code{p} and \code{q} denoting flow (e.g., a
direct assignment) from variable \code{p} to variable \code{q}.  Online
cycle elimination by F\"{a}ndrich et al.
\cite{pldi:1998:Fahndrich} detects cycles in the
constraint graph and collapses all nodes in a cycle into a
representative node, since such nodes will have identical points-to
information. The technique of Nasre
\cite{ismm:2012:Nasre} extends such constraint graph
reasoning based on the observation that if two nodes have the same
dominator in the constraint graph, then they are clones: the values
flowing to them are (only) those of the dominator node. Several other
constraint graph optimizations are applied off-line (i.e., before the
points-to analysis runs).  Prime examples of such techniques are
Rountev and Chandra's \cite{pldi:2000:Rountev} and Hardekopf and Lin's
\cite{sas:2007:Hardekopf}. (Hardekopf and Lin have also applied similar
ideas in a hybrid online/offline setting \cite{pldi:2007:Hardekopf}.)  Both
of these techniques perform an off-line detection of equivalent
points-to sets and use this knowledge to eliminate redundant work in
subsequent points-to computations. Our data structure can be seen as
somewhat analogous to constraint-graph techniques, in the sense that
we do not compute the flow of objects or the fully expanded set of all
possible alias pairs. Instead, we compute the ``wiring'' (i.e., the
alias relationships, locally, that the program induces) and keep the
alias information in condensed form, until it needs to be queried by a
client analysis.

Another conceptual relative of our data structure is the model
presented by Madhavan et al. \cite{article:2015:Madhavan} for modular \emph{may}
analyses. That model is similar in that it invents abstract nodes for
heap objects that resemble ours (without the equivalence-class
nature). The Madhavan et al. approach aims to achieve modular
reasoning, i.e., to model the heap effects of a method without knowing
its calling environment. To do so, the approach creates abstract
nodes that represent concepts such as ``whichever object variable
\code{x} may point to''. Our data structure has nodes with a similar
meaning, however we also take advantage of the ``must'' nature of
the analysis to merge nodes, every time the same access path can
reach both.


\paragraph{Must-Analyses for Aliasing.}
There are several instances of past work that apply must reasoning in
pointer analysis. These mostly serve to paint the landscape of
potential applicability of our data structure.

Ma et al.~\cite{isola:2008:Ma} present an algorithm for
null-pointer dereference detection using a context-insensitive
may-alias and a must-alias analysis; the latter is used to increase
the precision of the former, by enabling strong updates when possible.

Nikoli\'{c} and Spoto~\cite{ictac:2012:Nikolic} present a
must-alias analysis that tracks aliases between program expressions
and local variables (or stack locations, since they analyze Java
bytecode, which is a stack-based representation).  The analysis is a
generator of constraints, which are subsequently solved to produce the
analysis results.
% Abstractly,
%this is a relative of our Datalog-based approach, but it is unclear
%how the two may compare in terms of engineering tradeoffs.

Hind et al. \cite{article:1999:Hind} present a collection of pointer
analysis algorithms. Among them, the most relevant to this work is a
flow-sensitive interprocedural pointer alias analysis. The authors
optimistically produce \emph{must} information for pointers to single
non-summary objects.

Emami et al.~\cite{pldi:1994:Emami} present an approach that
simultaneously calculates both must- and may-point-to information for
a C analysis. Their empirical results ``show the existence of a
substantial number of definite points-to relationships, which forms
very valuable information''---much in line with our own
experience.

Must- information is often computed in conjunction with a
client analysis. One of the best examples is the typestate
verification of Fink et al.~\cite{issta:2006:Fink}, which demonstrates the value of
a must-analysis and the techniques that enable it.

%% The analysis of \cite{ecoop:2012:De} is essentially a
%% flow-sensitive may-point-to analysis that performs strong updates, as
%% it maps \emph{access paths} to \emph{heap objects} (abstracted by
%% their allocation sites). 
%% The approach uses a flow-insensitive
%% may-point-to analysis to bootstrap the main analysis. However, it
%% provides no \emph{definite} knowledge of any sort, since the aim is to
%% increase the precision of the may-analysis. For instance, even if an
%% access path points to a single heap object, according to the De and
%% D'Souza analysis, there is no \emph{must} point-to information
%% derived, since this object could be a summary object (i.e., one that
%% abstracts many objects allocated at the same allocation site). To
%% reason about such cases, other approaches, such as the more expensive
%% shape analysis algorithms \cite{article:2002:Sagiv},
%% additionally maintain summary information per heap object. In this
%% way, they allow must point-to edges to exist only if the target is
%% definitely not a summary node.

%An alternative approach to handle this shortcoming is to propagate
An approach for integrating \emph{must} point-to reasoning in an
analysis is to propagate such
%\emph{must}-point-to 
information only at instructions where we know that the given heap
allocation target still refers to the last object allocated at that
site \cite{popl:1995:Altucher}. Thus, an execution path
that may create another object at the same site (such as when reaching
the end of the loop) would invalidate any previous must-point-to facts
(i.e., it will stop them from propagating any further).

Generally, must-analyses can vary greatly in sophistication and can be
employed in an array of different combinations with may-analyses.  The
analysis of Balakrishnan and Reps~\cite{sas:2006:Balakrishnan}, which
introduces the \emph{recency abstraction}, distinguishes between the
most recently allocated object at an allocation site (a concrete
object, allowing strong updates) and earlier-allocated objects
(represented as a summary node). The analysis additionally keeps
information on the size of the set of objects represented by a summary
node. At the extreme, one can find full-blown shape analysis
approaches, such as that of Sagiv et
al.~\cite{article:2002:Sagiv}, which explicitly maintains must-
and may- information simultaneously, by means of three-valued truth
values, in full detail up to predicate abstraction: a
relationship can definitely hold (``must''), definitely not hold
(``must not'', i.e., negation of ``may''), or possibly hold
(``may''). Summary and concrete nodes are again used to represent
knowledge, albeit in full detail, as captured by arbitrary predicates
whose value is maintained across program statements, at the cost of a
super-exponential worst-case complexity.

Jagannathan et al.~\cite{popl:1998:Jagannathan} present
an algorithm for must-alias analysis of functional languages. The
algorithm adapts must-alias insights to the setting of captured
variables.
%captured in closures.  
For instance, must-alias information for
non-summary objects permits strong updates, which the authors find to
improve analysis precision. We employ must-alias analysis results
quite similarly in applications of our model analysis.



%Hardekopf and
%Lin's approach is impressively general, computing hash codes that
%encode all the logical processing of a points-to set that is induced
%by the current program and, thus, detecting equivalent points-to sets
%even through complex program patterns.

% \citep{Heintze:2001:UAA:378795.378855} ; pre-transitive constraint-graph and reachability caching


\section{Defensive Analysis}

There is certainly past work that attempt to ensure a sound
whole-program analysis, but none matches the generality and
applicability of our approach. We selectively discuss representative
approaches.

The standard past approach to soundness for a careful static analysis
has been to ``bail out'': the analysis detects whether there are
program features that it does not handle soundly, and issues warnings,
or refuses to produce answers. This is a common pattern in
abstract-interpretation~\cite{popl:1977:Cousot} analyses,
such as Astr\'{e}e~\cite{sas:2007:Delmas}, which have traditionally
emphasized sound handling of conventional language features. However,
this is far from a solution to the problem of being sound for opaque
code: refusing to handle the vast majority of realistic programs can
be argued to be sound, but is not usefully so. In contrast, our work
handles \emph{all} realistic programs, but returns partial (but sound)
results, i.e., produces non-empty points-to sets for a subset of the
variables. It is an experimental question to determine whether
this subset is usefully large, as we do in our evaluation.

%\citet{ecoop:2004:Hirzel,article:2007:Hirzel}
Hirzel et al. \cite{ecoop:2004:Hirzel,article:2007:Hirzel} use an
online pointer analysis to deal with reflection and dynamic loading by
monitoring their run-time occurrence, recording their results, and
running the analysis again, incrementally. However, this is hardly a
\emph{static} analysis and its cost is prohibitive for precise
(context-sensitive) analyses, if applied to all reflection
actions. 

%\citet{pldi:2007:Lattner}
Lattner et al. \cite{pldi:2007:Lattner} offer an algorithm that can apply to incomplete
programs, but it assumes that the linker can know all callers (i.e.,
there is no reflection---the analysis is for C/C++) and the approach
is closely tied to a specific flow-insensitive, unification-based
analysis logic~\cite{popl:1996:Steensgaard}, necessary
for simultaneously computing inter-related points-to, may-alias, and
may-escape information.

%\citet{pldi:2000:Sreedhar}
Sreedhar et al. \cite{pldi:2000:Sreedhar} present the
only past approach to explicitly target dynamic class loading,
although only for a specific client analysis (call
specialization). Still, that work ends up making many statically
unsound assumptions (requiring, at the very least, programmer
intervention), illustrating well the difficulty of the problem, if not
addressed defensively. The approach assumes that only the public API
of a ``closed world'' is callable, thus ignoring many uses of
reflection. (With reflection, any method is callable from unknown
code, and any field is accessible.) It ``[does] not address the Java
features of reloading and the Java Native Interface''. It
``optimistically assumes'' that ``[the extant state of statically
  known objects] remains unchanged when they become reachable from
static reference variables''. It is not clear whether the technique is
conservative relative to adversarial native code (in system libraries,
since the JNI is ignored). Finally, the approach assumes the existence
of a sound may-point-to analysis, even though none exists in practice!
% (i.e., the points-to analysis may miss
%objects when asked ``what objects can this variable point to'', as it is
%in the Sreedhar et al. paper).

Traditional conservative call-graph construction (\emph{Class
  Hierarchy Analysis (CHA)} \cite{ecoop:1995:Dean} or
\emph{Rapid Type Analysis (RTA)} \cite{oopsla:1996:Bacon}) is
unsound.  Such algorithms explore the entire class hierarchy for
matching (overriding) methods and consider all of them to be potential
virtual call targets. However, even this is not sufficient for a sound static
analysis of opaque code: classes can be generated and loaded
dynamically during program execution. CHA cannot find target methods
that do not even exist statically, yet modeling them is precisely what
is needed for soundness in real-world conditions. For instance, Java
applications, especially in the enterprise (server-side) space, employ
dynamic loading heavily, and patterns such as \emph{dynamic proxies}
have been standardized and used widely since the early Java days.

Furthermore, such heuristic ``best-effort'' over-approximation is
detrimental to analysis precision and performance. CHA is an example
of a loose over-approximation in an effort to capture most dynamic
behaviors.
%(Similar loose over-approximations have been proposed, for
%instance, for reflection analysis~\cite{aplas:2015:Smaragdakis}.) 
Loose over-approximations compute many more possible targets than
those that realistically arise. This yields vast points-to sets that
render the analysis heavyweight and useless due to imprecision.
(Avoiding such costs is exactly why past analyses have often opted for glaringly
unsound handling of opaque code features.)
Our lazy representation of ``don't know''/''cannot bound'' values as empty
sets addresses the problem, by keeping all points-to sets compact.


The conventional handling of reflection in may-point-to analysis
algorithms for
Java~\cite{www:wala-reflection,ecoop:2014:Li,aplas:2005:Livshits,thesis:Livshits,aplas:2015:Smaragdakis,sas:2015:Li}
is unsound, instead relying on a ``best-effort'' approach.  Such past
analyses attempt to statically model the result of reflection
operations, e.g., by computing a superset of the strings that can be
used as arguments to a \code{Class.forName} operation (which accepts a
name string and returns a reflection object representing the class
with that name).
The analyses are unsound when faced with a completely unknown string:
instead of assuming that \emph{any} class object can be returned, the
analysis assumes that \emph{none} can. The reason is that
over-approximation (assuming any object is returned) would be
detrimental to the analysis performance and precision. Even with an
unsound approach, current algorithms are heavily burdened by the use
of reflection analysis. For instance, the documentation of the
\textsc{Wala} library directly blames reflection analysis for
scalability shortcomings
\cite{www:wala-reflection},\footnote{The \textsc{Wala} documentation
  is explicit: ``\emph{Reflection usage and the
    size of modern libraries/frameworks make it very difficult to
    scale flow-insensitive points-to analysis to modern Java
    programs. For example, with default settings, WALA's pointer
    analyses cannot handle any program linked against the Java 6
    standard libraries, due to extensive reflection in the
    libraries.}''~\cite{www:wala-reflection}} and enabling reflection
on the \textsc{Doop} framework slows it down by an order of magnitude
on standard benchmarks~\cite{aplas:2015:Smaragdakis}.
% SPACE
Furthermore, none
of these approaches attempt to model dynamic loading---a ubiquitous
feature in Java enterprise applications.

%Most importantly,
%however, no such approach can possibly ever be sound purely
%statically. Even if the analysis could feasibly assume that \emph{any}
%class object is returned by a \code{forName} operation, this
%%over-approximation 
%cannot account for dynamically generated and loaded
%classes---a ubiquitous feature in Java enterprise
%applications.%