\chapter{Must-Alias Analysis: Logical Model}
\label{chapter:must-logic}
\epigraph{Logic, my dear Zoe, merely enables one to be wrong with authority.}{\textit{The 2nd Doctor} - Doctor Who}

As previously mention at the beginning of Chapter~\ref{chapter:intro}, alias analysis is closely related to pointer analysis with the main difference being that the main goal of alias analysis is finding aliasing among program expressions whereas the main goal of pointer analysis is to reason about the objects that program expressions may point to. In both case, one can be employed to support the reasoning of the other.

The vast majority of pointer analysis techniques that have appeared in the recent research literature (e.g., \cite{pldi:1999:Yong,pldi:2003:Berndl,cc:2013:Kastrinis,pldi:2007:Hardekopf,article:2005:Milanova,pldi:2004:Whaley,pldi:2006:Sridharan}) are \emph{may}-analyses. That is, the techniques attempt to \emph{over}-approximate an unattainable, fully precise result (recall Section~\ref{sec:intro:may-must}). All possible aliasing expressions are guaranteed to be included in the outcome of a may-alias analysis. All possible abstract objects that may be referenced by a variable are included in the variable's may-points-to set. However, spurious inferences---which will never occur in program execution---may also be included in the analysis output.

In contrast, an \emph{under}-approximate, \emph{must}-analysis (or \emph{definite}-analysis) is often desirable. A must-analysis computes aliasing or points-to relationships that are guaranteed to always hold during program execution, at the cost of missing some inferences. In practice, a must-alias analysis is more probable than a must-points-to analysis. Aliasing relationships are more generally provable from local inspection of the program text, whereas points-to facts are harder to establish in a conservative must- fashion. The difficulty is dual: First, the allocation sites of objects are often far from their use sites, making the establishment of must-points-to relationships unlikely. Second, must-points-to reasoning requires careful modeling of abstract vs. concrete objects. For instance, for techniques such as \emph{strong updates} (i.e., replacing the value of an object field at a store instruction) it is not sufficient to know the abstract object that the base expression must-points-to, since the abstract object may conflate many concrete objects during program execution (e.g., in a loop), and only one of them will have its field updated.

A must-alias analysis is typically flow-sensitive, i.e., it computes information per-program-point, respecting the control-flow of the program (recall Section~\ref{sec:intro:flowSensitivity}). It offers several applications:

\begin{itemize}
\item It is useful for optimizations---e.g., constant folding, common subexpression elimination, and register allocation;

\item It can increase the precision of bug detectors (that traditionally have a high false-warnings rate): Nikoli\'{c} and Spoto~\cite{ictac:2012:Nikolic} report that a must-alias analysis significantly increases the precision of both a null-reference detector (46\% fewer warnings) and a non-termination detector (11\% fewer warnings). Earlier work has reported similar benefits~\cite{isola:2008:Ma};

\item It can be used as an internal component as part of a more complex analysis. For instance, must-alias results may enable an analysis to perform ``strong updates'' at instructions that modify the heap. Earlier work has used must-alias analysis to similar benefit~\cite{pldi:1994:Emami,popl:1998:Jagannathan};

\item It can be invaluable for better program understanding; the results of a must-inference are guaranteed facts, of immediate value to the human programmer.
\end{itemize}

To illustrate must-alias reasoning, consider the code example in Figure~\ref{fig:must-logic:snippet}. Even at this size, inspecting the program requires human effort. The output consists of must-alias pairs: expressions that are guaranteed to point to the same object---denoted by ``$\sim$''. In this example, \code{a2.next} and \code{a1} form an alias pair after line 8. (Other alias pairs include \{\code{a1.next} $\sim$ \code{null}\} after line 7, \{\code{a2.next.next} $\sim$ \code{a2}\} after line 11, and more.) Alias pairs are established by direct variable assignments---which are plentiful in a compiler intermediate language, although less so in original source code---as well as heap stores and loads. Other aliasing relationships hold throughout the program. Establishing them often requires some inter-procedural reasoning---e.g., to see the aliasing effects of the constructor call on lines 8, 9, or 10. Constructors feature prominently in the example, since they are one of the best sources of must-alias information in a typical program.

A \emph{must}-alias analysis has to report aliases only when they are guaranteed to hold, and needs to invalidate them on store instructions or method calls that may change the fields of objects pointed by sub-expressions in an alias pair.

\begin{figure}[htb!p]
\begin{javacode}
class Node {
	Node next;
	Node(Node next) { this.next = next; }
	void wrap() { next.next = this; }
}

void main() {
	Node a1 = new Node(null);
	Node a2 = new Node(a1);
	Node a3 = new Node(null);
	a1.next = a3;
	a2.wrap();
}
\end{javacode}
\caption{Code snippet for illustrating must-alias reasoning.}
\label{fig:must-logic:snippet}
\end{figure}

For instance, line 10 invalidates the alias pair \{\code{a1.next} $\sim$ \code{null}\}---regardless of whether new alias pairs are established, via inter-procedural reasoning. However, the analysis is sound (i.e., it remains a must-analysis---recall Section~\ref{sec:intro:soundness}) if it also invalidates alias pairs for expressions involving \code{a2.next} or \code{a3.next}. The base specification of a must-alias analysis has to integrate such soundness safeguards, while interplay with other analyses (e.g., a may-not-alias analysis) can lead to more inferences.

This chapter presents a simple declarative model of a must-alias analysis over access paths (i.e., expressions of the form ``\texttt{\args{var}(\args{.fld})}*''). The model underlies the implementation of must-alias analysis in \doop{}, in which must-alias analysis is employed as an enhancer of its standard array of may-analyses (e.g., in order to enable ``strong updates''). The model is interesting in a few different ways:

\begin{itemize}
\item It is an instance of a flow-sensitive analysis in Datalog. As such, it introduces idioms and patterns also used in a multitude of other (current or future) analyses in \doop{}.

\item The analysis is minimal, yet models the core features of a general must-alias analysis in a handful of declarative rules.\footnote{The analysis core presented here is the basis of a much larger, full-fledged (over 300 Datalog rules) must-alias analysis implementation in \doop{}.} In this way, the analysis semantics are easily understood and can be further enhanced. The rules allow configurability and employ several techniques for conciseness and power.

\item The use of context, in particular, is crucial: the analysis applies context variables, much like in traditional may analyses (e.g., \cite{pldi:2013:Kastrinis,article:2015:Smaragdakis} and what was presented in previous chapters), yet uses the context highly unconventionally. Context is used as ``fuel'', to guarantee the ``must'' nature of the analysis: must-alias inferences are propagated inter-procedurally, with context extended for every call. When maximum context depth is reached, inferences cannot propagate any further.

\item A major benefit of the analysis is its \emph{incrementality}. In a well-specified must-alias analysis, soundness is not compromised if only a portion of the program-under-analysis or its libraries are available. This key element is emphasized in our declarative model. We control the program points where the full analysis applies and leverage context-sensitivity to allow analysis of other program points. In essence, our analysis infers normal alias pairs for a user-selected core part of the program, and infers conditional, context-qualified alias pairs for other parts that interact with the core program.

\item The analysis gives rise to several observations, concerning the representation of equivalence relations in a Datalog engine, and the need for implicit encoding of aliasing.

\end{itemize}


\section{Logical Model}
\label{sec:must-logic:model}

We demonstrate a minimal Datalog model of an inter-procedural must-alias analysis via changes to the existing model for context-sensitive, flow-insensitive points-to analysis algorithms presented in Section~\ref{sec:intro:model}. As in previous chapters, the logical formalism of this model is very close to the core components of our actual analysis implementation.

But contrary to previous chapters, we need to model a flow-\emph{sensitive} analysis and this dictates certain alterations on our input language. Mainly, we assume a static-single assignment (SSA) form on our intermediate representation. Recall from Section~\ref{sec:intro:ssa} that every local variable is assigned exactly once and for variables with multiple assignments in the original source code, the merging of their values is indicated by a \emph{phi-node} in the SSA representation. For the purposes of static analyses that do not track path conditions, it is not relevant which of the two (or more) values is actually picked, but it is important (for flow-sensitive variants) that such merging points are indeed modeled accordingly.


\paragraphhead{Input domain.}
Figure~\ref{fig:must-logic:input-domain} demonstrates the needed alterations in the domain of our input language (presented in Figure~\ref{fig:intro:input-domain}).

\begin{figure}[b!htp]
\begin{tabular}{l}
\args{A} is an access path of the form \args{V}.(\args{F})* \\
\end{tabular}
\caption[]{Additions to the domain of our input intermediate language.}
\label{fig:must-logic:input-domain}
\end{figure}


\paragraphhead{Input relations.}
Our input relations, shown in Figure~\ref{fig:must-logic:input}, are similar to those presented in Figure~\ref{fig:intro:input} with the main difference that they also encode of the actual instruction they represent, in order to enable a flow-sensitive reasoning. Regarding invocations, we mainly focus on virtual calls (simplified to \relname{Call} instead of \relname{VCall}) and assume a program in a single-return form, for each method. In addition, the \relname{Phi} relation captures phi-node instructions, where values of multiple ``\args{from}'' vars merge into local var \args{to}. The \relname{Next} relation expresses directed edges in the control-flow graph (CFG) meaning that instruction \args{j} is a successor of instruction \args{i}. Relation \relname{InMethod} encodes the obvious semantics of method \args{meth} containing instruction \args{i}.

The last two input relations are somewhat more advanced. \altrelname{Resolved} is a predicate that can be computed by an external call-graph or may-point-to analysis: it holds variables that are determined to only point to objects with a unique dynamic type, so that virtual method calls are resolved. (Note that the form of the predicate is context-insensitive, yet the analysis that computes it may be context-sensitive, for increased precision---the contexts are merely projected out.) Finally, \altrelname{RootMethod} is a predicate over methods, in order to start must-alias reasoning from a user-selected set of methods. As will be explained bellow, our analysis algorithm will venture beyond these root methods only to the extent that its context constructor allows.

\begin{figure}[t!hbp]
\begin{tabular}{l l}
\rel{Move}{i: I, to: V, from: V}             & \comm{// i: to = from} \\
\rel{Load}{i: I, to: V, base: V, fld: F}     & \comm{// i: to = base.fld} \\
\rel{Store}{i: I, base: V, fld: F, from: V}  & \comm{// i: base.fld = from} \\
\rel{Call}{i: I, base: V, sig: S}            & \comm{// i: base.sig(\ldots)} \\
\rel{FormalReturn}{i: I, meth: M, ret: V}    & \comm{// i: return ret;} \\
\\
\rel{Phi}{i: I, to: V, from1: V, \ldots}     & \comm{// i: to = $\phi$(from1, \ldots)} \\
\rel{Next}{i: I, j: I} \\
\rel{InMethod}{i: I, meth: M} \\
\\
\altrel{Resolved}{var: V, type: T} \\
\altrel{RootMethod}{meth: M} \\
\end{tabular}
\caption[]{The (altered) input Datalog relations describing the program under analysis.}
\label{fig:must-logic:input}
\end{figure}


\paragraphhead{Computed (output) relations.}

Figure~\ref{fig:must-logic:output} shows the computed relations of our must-alias analysis. The first relation, \relname{MustAlias}, is the main output of the analysis and is defined on access paths. The semantics are that access path \args{ap1} aliases access path \args{ap2} (i.e., they are guaranteed to point to the same heap object, or to both be \code{null}) right after program instruction \args{i}, executed under context \args{ctx}, provided that the instruction is indeed executed under \args{ctx} at program run-time. The two access paths are said to form an \emph{alias pair}.  
 
The other main computed relation represents intermediate results of the analysis. Relation \relname{MustCallGraphEdge} holds information for fully-resolved virtual calls: invocation site \args{invo} will call method \args{meth} under the given (caller and collee) contexts.

\begin{figure}[hp]
\begin{datalog}
\rel{MustAlias}{i: I, ctx: C, ap1: A, ap2: A} \\
\rel{MustCallGraphEdge}{invo: I, callerCtx: C, meth: M, calleeCtx: C} \\
%
\noindent\rule{\textwidth}{0.5pt}\\
%
\cons{AP}{access path expression}{?ap: A} \\
\cons{PrimeAP}{ap: A}{?newAp: A} \\
\cons{UnprimeAP}{ap: A}{?newAp: A} \\
\cons{NewContext}{invo: I, ctx: C}{?newCtx: C}
\end{datalog}
\caption[]{The core Datalog output relations and constructors of access paths and contexts.}
\label{fig:must-logic:output}
\end{figure}


\paragraphhead{Constructors.}

We assume a constructor function \consname{AP} that produces access paths. For instance, the expression ``\cons{AP}{var.fld1.fld2}{?ap}'' means that the access path \args{?ap} has length 3 and its elements are given by the values of bound logical variables \args{var}, \args{fld1} and \args{fld2}. We shall also use \consname{AP} as a pattern matcher over access paths. For instance, the expression ``\cons{AP}{\_.fld}{ap}'' binds the value of logical variable \args{fld} to the last field of access path \args{ap}. (\args{\_} is an anonymous variable that can match any value.)

Additionally, we manipulate access paths with two functions, \consname{PrimeAP} and \consname{UnprimeAP}. \consname{PrimeAP} takes an access path and returns a new one by ``priming'' the base variable of the original. \consname{UnprimeAP} reverses this mapping. For instance, \cons{PrimeAP}{``v.fld''}{``v\textbf{'}.fld''}. \consname{UnprimeAP} only applies to access paths with primed variables as their base---otherwise the rule fails to match. Priming and unpriming of access paths is done at method call and return sites, to mark access paths that arrive from callers. This is necessary for avoiding confusion of variables in recursive calls.

Similarly, we construct new contexts using function \consname{NewContext}. The definition of this constructor serves to configure the analysis for different context settings, as discussed later. If \consname{NewContext} does not return a value (e.g., because the maximum context depth has been reached), the current rule employing the constructor will not produce facts. The constant \ctxAll{} is used to signify the initial context.

Constructors of access paths and contexts are much like other relations. In practical analyses, the space of access paths and contexts is made finite, by bounding their length. Therefore, all possible access paths and contexts could be computed prior to the analysis start and supplied as inputs. However, this is unlikely to be desirable in practice, for efficiency reasons, and is limiting in principle: by separating constructors, our model also allows analyses with unbounded access paths and contexts.


\section{Analysis Logic}

We have broken down our analysis model in four groups of rules. For conciseness, we have employed some syntactic sugar (also recall Section~\ref{sec:intro:datalog} regarding disjunction and negation), which straightforwardly maps to more complex Datalog rules:

\begin{itemize}
\item We use the shorthand \relname{P*} for the reflexive, symmetric, transitive closure of relation \relname{P}, which is assumed to be binary. For larger arities, underscore (\_) variables are used to distinguish variables of a relation that are affected by the closure rule. Specifically, \rel{MustAlias*}{i, ctx, \_, \_} denotes the reflexive, symmetric, transitive closure of relation \relname{MustAlias} with respect to its last two variables.

\item We introduce a for-all syntactic sugar that hides a Datalog pattern for enumerating all members of a set and ensuring that a condition holds universally.\footnote{Emulating universal quantification in Datalog requires ordered domains. In practice, this is not a restriction. An arbitrary ordering relation (e.g., by internal id of input facts as assigned by the implementation) can be imposed on all our domains.} An expression ``\dlforall{} \args{i}: \rel{P}{i} \dlThen{} \rel{Q}{i, \ldots}'' is true if \rel{Q}{i, \ldots} holds for all \args{i} such that \rel{P}{i} holds. Such an expression can be used in a rule body, as a condition for the rule's firing. Multiple variables can be quantified by a \dlforall{}. Variables not bound remain implicitly existentially quantified, as in conventional Datalog. However, the existential quantifier is interpreted as being outside the universal one. For instance, ``\dlforall{} \args{i, j}: \rel{P}{i, j, k} \dlThen{} \rel{Q}{i, j, k, l}'' is interpreted as ``there exist \args{k, l} such that for all \args{i, j} \ldots''.
\end{itemize}


\paragraphhead{Base Rules.}
Figure~\ref{fig:must-logic:baserules} lists six rules: one to initialize interesting analysis contexts and five for must-alias inferences. The former rule employs configuration predicate \altrelname{RootMethod}. This predicate designates methods that are to be analyzed unconditionally: the inference is made under the special context value \ctxAll{}. For a non-root method, aliasing inferences can only be made under a specific context, for which the method has been computed to be reachable. The results can then be used by the caller that produced that context. They are not, however, established as unconditional results (in an \ctxAll{} context), which would be usable independently.

The above mechanism controls the application extent of the analysis. Recall that incrementality is a key benefit of a must-alias analysis. Therefore, it is desirable to be able to apply the algorithm as locally as the user may desire. The context mechanism is then used to explore other code, but only to the extent that such exploration benefits the root methods intended for analysis.

The next four \relname{MustAlias} rules handle one instruction kind each: \relname{Move}, \relname{Phi}, \relname{Load}, and \relname{Store}. The \relname{Move} rule merely establishes an aliasing relationship between the two assigned variables, at the point of the move instruction. The \relname{Phi} rule promotes aliasing relationships that hold for all the right-hand sides of a phi-node instruction to its left hand side. The \relname{Load} and \relname{Store} rules establish aliases between the loaded/stored expression, \args{base.fld}, and the local variable used. Finally, the last \relname{MustAlias} rule makes the \relname{MustAlias} relation symmetrically and transitively closed.

\begin{figure}[htp]
\begin{datalog}
\rel{Reachable}{meth, \ctxAll{}} \dlIf{} \altrel{RootMethod}{meth}. \\
\\
\rel{MustAlias}{i, ctx, \consapp{AP}{from}, \consapp{AP}{to}} \dlIf{} \\
    \rel{Move}{i, to, from}, \\
    \rel{InMethod}{i, meth}, \rel{Reachable}{meth, ctx}. \\
\\
\rel{MustAlias}{i, ctx, ap, \consname{AP}(\args{to})} \dlIf{} \\
    (\dlforall{} \args{from}: \rel{Phi}{i, to, \ldots, from, \ldots} \dlThen{} \rel{MustAlias}{i, ctx, \consapp{AP}{from}, ap}), \\
    \rel{InMethod}{i, meth}, \rel{Reachable}{meth, ctx}. \\
\\
\rel{MustAlias}{i, ctx, \consapp{AP}{to}, \consname{AP}(\args{base.fld})} \dlIf{} \\
    \rel{Load}{i, to, base, fld}, \\
    \rel{InMethod}{i, meth}, \rel{Reachable}{meth, ctx}. \\
\\
\rel{MustAlias}{i, ctx, \consapp{AP}{from}, \consname{AP}(\args{base.fld})} \dlIf{} \\
    \rel{Store}{i, base, fld, from}, \\
    \rel{InMethod}{i, meth}, \rel{Reachable}{meth, ctx}. \\
\\
\rel{MustAlias}{i, ctx, \_, \_} \dlIf{} \rel{MustAlias*}{i, ctx, \_, \_}.
\end{datalog}
\caption[]{The core, base Datalog rules for a model must-alias analysis.}
\label{fig:must-logic:baserules}
\end{figure}


\paragraphhead{Inter-Procedural Propagation Rules.}
Figure~\ref{fig:must-logic:propagationrules} presents four rules responsible for the inter-procedural propagation of access path aliasing.

The first rule continues the handling of program instructions with a treatment of \relname{Call}. At a \relname{Call} instruction, for method signature \args{sig} over receiver \args{base}, if \args{base} has a unique (resolved) type, then the method is looked up in that type, a \relname{MustCallGraphEdge} is inferred from the invocation instruction to the target method and the method is also marked as \relname{Reachable} with a callee context computed using constructor \consname{NewContext}. Recall that the \consname{NewContext} function may fail to return a new context (e.g., because \args{ctx} has already reached the maximum depth and \args{calleeCtx} would exceed it) in which case the rule will not infer new facts.

The other three rules handle aliasing induced at a method invocation site. Despite their rather daunting form, the rules are quite straightforward. The first states that, at the first instruction of a called method, the formal and actual arguments are aliased. In combination with other rules (discussed next, under ``Access Path Extension'') this is sufficient for transferring all alias pairs from the caller to the callee. The actual argument is ``primed'' appropriately, to mark that it is received from a caller. For instance, if the analyzed program contains a call ``\code{foo(x)}'' to a method defined as ``\code{void foo(Object y)}'', the rule will simply infer that \code{x}' and \code{y} are aliased. The rule infers the same aliasing for the base variable of the method call and the pseudo-variable \code{this} inside the receiver method. (Note how the first instruction of the called method is computed as the only instruction in the method that has no CFG predecessors: \dlforall{} \args{k} \dlThen{} !\rel{Next}{k, firstInstr}.  This convention is assumed to hold for our input intermediate language.)

The third rule similarly identifies the first instruction of a called method. It then propagates to it all alias pairs that hold \emph{after all predecessor instructions}, \args{j}, of the calling instruction, \args{i}. The base variables of the alias pairs are ``primed'', as appropriate, to denote that they come from the caller.

The fourth and final rule performs the inverse mapping of access paths from a return instruction to the call site. For alias pairs to propagate back (to the caller, with context \args{callerCtx}), they need to hold either in the appropriate context (\args{calleeCtx}, which matches \args{callerCtx} in the call graph), or unconditionally, i.e., with context \ctxAll{}. Access paths are ``unprimed'' when propagating to the caller. Note that this implies that local alias pairs (e.g., among local variables of the callee) do not propagate to the caller.

Crucially, the handling of a method return is the \emph{only} point where a context can become stronger. \relname{MustAlias} facts that were inferred to hold under the more specific \args{calleeCtx} are now established, modulo unpriming, under \args{callerCtx}.

\begin{figure}[thp]
\begin{datalog}
\rel{MustCallGraphEdge}{i, callerCtx, toMeth, ?calleeCtx}, \\
\rel{Reachable}{?calleeCtx, toMeth} \dlIf{} \\
    \rel{Call}{i, base, sig}, \\
    \rel{InMethod}{i, inMeth}, \rel{Reachable}{inMeth, callerctx}, \\ 
    \altrel{Resolved}{base, type}, \rel{LookUp}{type, sig, toMeth}, \\
    \cons{NewContext}{i, ctx}{?calleeCtx}. \\
\\
\rel{MustAlias}{firstInstr, calleeCtx, ?ap1, ?ap2} \dlIf{} \\
    \rel{MustCallGraphEdge}{i, \_, toMeth, calleeCtx}, \\
    \rel{InMethod}{firstInstr, toMeth}, (\dlforall{} \args{k} \dlIf{} !\rel{Next}{k, firstInstr}), \\
    ((\rel{FormalArg}{toMeth, n, toVar}, \rel{ActualArg}{i, n, fromVar}) ; \\
     (\rel{ThisVar}{toMeth, toVar}, \rel{Call}{i, fromVar, \_})), \\
    \cons{PrimeAP}{\consapp{AP}{fromVar}}{?ap1}, \cons{AP}{toVar}{?ap2}. \\
\\
\rel{MustAlias}{firstInstr, calleeCtx, ?ap1, ?ap2} \dlIf{} \\
    \rel{MustCallGraphEdge}{i, callerCtx, toMeth, calleeCtx}, \\
    \rel{InMethod}{firstInstr, toMeth}, (\dlforall{} \args{k} \dlIf{} !\rel{Next}{k, firstInstr}), \\
    (\dlforall{} \args{j}: \rel{Next}{j, i} \dlThen{} \rel{MustAlias}{j, callerCtx, callerAp1, callerAp2}), \\
    \cons{PrimeAP}{callerAp1}{?ap1}, \cons{PrimeAP}{callerAp2}{?ap2}. \\
\\
\rel{MustAlias}{i, callerCtx, ?ap1, ?ap2} \dlIf{} \\
    \rel{MustCallGraphEdge}{\_, callerCtx, toMeth, calleeCtx}, \\
    \rel{FormalRet}{retInstr, toMeth, \_}, \\
    (\rel{MustAlias}{retInstr, calleeCtx, calleeAp1, calleeAp2} ; \\
     \rel{MustAlias}{retInstr, \ctxAll{}, calleeAp1, calleeAp2}), \\
    \cons{UnprimeAP}{calleeAp1}{?ap1}, \cons{UnprimeAP}{calleeAp2}{?ap2}.
\end{datalog}
\caption[]{Datalog rules for inter-procedural propagation of alias pairs.}
\label{fig:must-logic:propagationrules}
\end{figure}


\paragraphhead{Access Path Extension.}
Figure~\ref{fig:must-logic:accesspathext} contains a straightforward, yet essential, rule. This rule allows access path extension: if two access paths alias, extending them by the same field suffix also produces aliases. It is important to note that the constructor \consname{AP} is not used in the head of the rule, thus the extended access paths are not generated but assumed to exist. Therefore, the rule does not spur infinite creation of access paths.

This powerful rule is responsible for much of the simplicity of our must-alias analysis specification. For instance, recall how earlier we handled the mapping of actual to formal method arguments quite simply: we merely added an alias between the (primed) actual argument variable and the formal argument. It is the access path extension rule that takes care of also generalizing this mapping to longer access paths whose base variable is the actual argument of the call.

\begin{figure}[htp]
\begin{datalog}
\rel{MustAlias}{i, ctx, ?ap3, ?ap4} \dlIf{} \\
    \rel{MustAlias}{i, ctx, ap1, ap2}, \\
    \cons{AP}{ap1.fld}{?ap3}, \cons{AP}{ap2.fld}{?ap4}.
\end{datalog}
\caption[]{Datalog rule for access path extension.}
\label{fig:must-logic:accesspathext}
\end{figure}


\paragraphhead{Frame Rules: From One Instruction To The Next.}
The rule in Figure~\ref{fig:must-logic:framerules} determines how must-alias facts can propagate from one instruction to its successors. The rule simply states that all aliases are propagated if the instruction is not a store or a call. (Because of SSA, access paths cannot be invalidated via move instructions.)

\begin{figure}[htp]
\begin{datalog}
\rel{MustAlias}{i, ctx, ap1, ap2} \dlIf{} \\
    !\rel{Store}{i, \_, \_, \_}, !\rel{Call}{i, \_, \_}, \\
    (\dlforall{} \args{j}: \rel{Next}{j, i} \dlThen{} \rel{MustAlias}{j, ctx, ap1, ap2}).
\end{datalog}
\caption[]{Datalog frame rule propagating alias pairs from one instruction to the next.}
\label{fig:must-logic:framerules}
\end{figure}


\paragraphhead{Comments.}
The model we just presented is carefully designed to encompass a minimal, highly-compact but usefully representative must-alias analysis. There are several extensions that can apply, but all of them are analogous to features shown. For instance, we are missing a rule for propagating back to the caller complex access paths (i.e., of length greater than 1) that are based on the formal return variable. Similarly, store or call instructions do not invalidate aliasing between local variables---an extra rule could allow further propagation. Furthermore, it is not always necessary for an alias pair to hold in all predecessors: it could hold in one and others may be dominated by the instruction and not invalidate the alias pair. Our actual implementation contains the handling of such cases, but these complexities do not affect the discussion of our model.

These rules liberally employ negation. They establish that must-alias facts are propagated if some disabling conditions do \emph{not} hold. Therefore, for a full-fledged analysis, the rules need to be enriched with more preconditions, to cover all different kinds of program instructions that may invalidate access paths.


\section{Discussion}

There are several parts of the model and its implementation that are worth emphasizing.

\paragraphhead{Access Path Creation.}
Our access path constructor, \consname{AP}, hides the details of the space of access paths and their construction. There are several different policies that an analysis can pick. In theory, we could up-front populate the entire combinatorial space of ``\texttt{\args{var}(\args{.fld})}*'') up to a certain depth. However, the large sizes of the domains of variables and fields make this prohibitive. An efficient way to create access paths lazily (also used in our full implementation) is to initially generate all primitive access paths (variables and variable-single-field combinations) that appear explicitly in the program text, and then close the set of access paths by employing the rule:

\begin{datalog}
\cons{AP}{ap2.fld}{?newAp} \dlIf{} \rel{MustAlias}{\_, \_, ap1, ap2}, \cons{AP}{ap1.fld}{\_}.
\end{datalog}

Note that one use of the constructor \consname{AP} is in the head of the rule (thus generating new access paths on the fly) and one in the body (checking that the access path already exists). That is, extended access paths (\args{base.field}) are generated only if their base access path is found to be aliased with another path, which already exists with the \args{field} suffix.


\paragraphhead{Context-Sensitivity in Must-Alias.}
The use of context in our must-alias analysis is subtle. Context in a pointer analysis is used to distinguish different dynamic execution flows when analyzing a method. That is, the same method gets analyzed once per each applicable context, under different information. The context effectively encodes different scenarios under which the method gets called, allowing more faithful analysis in the specialized setting of the context.

Yet, the use of context in a must-alias analysis has been explored very little in the past. The reason is that there is little benefit to be gained from specializing incoming information (i.e., alias pairs) for a must-alias analysis. Pairs of aliasing access paths already offer a symbolic summary of a function's behavior, so that there is less need to analyze the function separately for different contexts: the access paths can be merely returned to the caller and they will be specialized there. Our analysis model employs context to transmit alias pairs from a caller to a callee, yet qualify them with the context identifier to which they pertain. This enables producing more alias pairs, however, their validity is conditional on the context used.

Still, this conditional information can be used for further inferences. For instance, \relname{MustAlias} inferences could be combined with allocation instructions (allocation sites can be viewed as global access paths) in order to determine, when possible, which objects an access path must point to. In turn, this can inform virtual method resolution, which our current rules (Figure~\ref{fig:must-logic:propagationrules}) only perform via a may-analysis by use of relation \altrelname{Resolved}. In this way, specialized alias relations for a given context can result in more inferences (since a method call may now have a known target). These inferences can be propagated back to the caller, where they hold unconditionally. (Recall that the rules handling returns can remove access path assumptions.)

Generally, the use of a deeper context in a must-analysis can extend its reach (or recall as discussed in Section~\ref{sec:intro:precision-recall}), allowing \emph{more inferences}, i.e., a larger result, whereas deeper context in a may-analysis it results in \emph{more precision}, i.e., a smaller result. 

What can our context be, however? In previous chapters, or typical context-sensitive pointer analyses in the literature, a variety of context creation functions can be employed. There are context flavors such as call-site sensitivity \cite{col:1981:Sharir,thesis:Shivers}, object sensitivity \cite{issta:2002:Milanova,article:2005:Milanova}, or type sensitivity \cite{popl:2011:Smaragdakis}. Our \consname{NewContext} constructor (employed at method calls) could be set appropriately to produce such context variety. However, the current form of our rules restricts our options to call-site sensitivity, with potential extra information adding to, but not replacing, call sites. Given the signature of constructor \consname{NewContext} the assumption is that the new context produced uniquely identifies both invocation site \args{invo} and \emph{its} context, \args{ctx}. Effectively, if \consname{NewContext} produces a \args{?newCtx} at all, it can do little other than push \args{invo} onto \args{ctx} and return the result.

The analysis then propagates \relname{MustAlias} pairs from (all predecessors of) call site \args{invo} under context \args{ctx} to the first instruction of a called method, \args{toMeth}, under context \args{?newCtx}. Thus, \args{?newCtx} should be enough to establish that these inferences \emph{must} hold. There is no room for conflating information from multiple execution paths (i.e., callers and calling contexts).\footnote{One could imagine doing so under the premise that all such calling contexts agree on the aliases they establish at the beginning of the callee function. However, this is unlikely to arise often in practice.}

The requirement that \consapp{NewContext}{invo, ctx} produce contexts that uniquely identify both \args{invo} and \args{ctx} means that context can only grow from an original source in our analysis. Consider a set of three methods, \code{meth1}, \code{meth2}, and \code{meth3}, each calling the next. If we allow \consname{NewContext} to produce contexts that are stacks of invocation sites, \args{i}, each starting with \ctxAll{} and growing up to depth 2, then starting from \code{meth1} we will propagate its aliases to \code{meth2}, which will propagate the resulting combined aliases to \code{meth3}. The propagation will stop there, i.e., the aliases of \code{meth1} cannot influence inferences for callees of \code{meth3}. However, \code{meth3} (assuming it is included in the root methods) will itself also be analyzed with a context of \ctxAll{}, allowing its own aliases (independently derived from those of \code{meth1} or \code{meth2}) to be a source of a similar propagation.


\paragraphhead{Representation of Equivalence Classes.}
\label{sec:must-logic:equivalence}
Relation \relname{MustAlias} encodes equivalence classes on access paths. Datalog inherently has no such notion and any attempt to compute a must-alias relation has to explicitly encode all aliasing pairs. E.g., if variable \code{v1} is an alias for variable \code{v2}, and \code{v2} of variable \code{v3}, we have to explicitly record the following pairs: \{\code{v1} $\sim$ \code{v2}\}, \{\code{v2} $\sim$ \code{v1}\}, \{\code{v2} $\sim$ \code{v3}\}, \{\code{v3} $\sim$ \code{v2}\}, \{\code{v1} $\sim$ \code{v3}\}, \{\code{v3} $\sim$ \code{v1}\}. This effect is exacerbated for longer access paths.

In theory, this redundancy will greatly hinder performance. In practice, it is often affordable because of keeping access paths short and computing must-alias information where needed, due to the locality of such information. (Aliasing pairs rarely propagate much deeper than the point in code where they were first established.) The analysis is fully modular and can be applied to any subset of the program code. Still, future work should address this shortcoming in the general setting of Datalog computation of equivalence relations.


\section{Conclusions}

The literature on must-alias analyses is sparse and the distance of specification to implementation is typically large. In our literature survey we have not found a single must-alias analysis publication that concretely refers to another and shows how its approach differs. Thus, we believe that our declarative model can offer a reference point for future work.

We believe that our model is clear yet concrete enough to spur further development and a better understanding of the comparative features of different must-alias analysis algorithms. In practical terms, must-alias analysis is valuable and woefully under-exploited in the literature. Our experiments show concrete value for (human) program understanding and (automatic) optimization.