Η διατριβή αυτή ανήκει στον ευρύτερο τομέα της στατικής ανάλυσης προγραμμάτων, η οποία στοχεύει στον αυτόματο συμπερασμό των ιδιοτήτων που παρουσιάζει κάποιο πρόγραμμα, με βάση την εξέταση του πηγαίου του κώδικα (ή κάποιας αντίστοιχης ενδιάμεσης αναπαράστασης), αλλά δίχως να απαιτείται κάποια πραγματική εκτέλεση του. Η δουλειά μας στη διατριβή αυτή επικεντρώνεται σε μια μεγάλη υποκατηγορία της στατικής ανάλυσης, αυτή της ανάλυσης \emph{δεικτών}. Μία ανάλυση δεικτών στοχεύει στο να υπολογίσει τα σύνολα αντικειμένων στα οποία μπορεί να ``δείξει'' κάθε έκφραση του προγράμματος (π.χ. τοπική μεταβλητή, πεδίο, κτλ.) σε όλες τις πιθανές εκτελέσεις του.

Σαν αποτέλεσμα, κάθε πρακτικός αλγόριθμος που στοχεύει να παρέχει ουσιαστικά αποτελέσματα αναγκάζεται να κάνει έναν πρώτο συμβιβασμό: χρειάζεται να κατασκευάσει κάποιο \emph{αφηρημένο} μοντέλο της μνήμης, όπου \emph{εικονικά} αντικείμενα αναπαριστούν (μία ή περισσότερες) \emph{διακριτές} δεσμεύσεις πραγματικών αντικειμένων. Ένα κλασικό παράδειγμα αυτού είναι η δέσμευση αντικειμένων από κάποια εντολή μέσα σε μία δομή επανάληψης. Η συνηθισμένη αντιμετώπιση \mbox{από} κάποιον αλγόριθμο ανάλυσης δεικτών είναι να θεωρηθούν όλα τα αντικείμενα που εν δυνάμει θα δεσμευτούν από την ίδια εντολή, σαν ένα μοναδικό, αφηρημένο αντικείμενο. Αυτό αποτελεί μία (από τις πολλές) πήγη ανακρίβειας στα αποτελέσματα της όποιας ανάλυσης. Ταυτόχρονα όμως, συμβιβασμοί σαν αυτόν, αν και οδηγούν σε εκτιμήσεις της συμπεριφοράς ενός προγράμματος και όχι σε απόλυτα αποτελέσματα, επιτρέπουν στις αναλύσεις να κάνουν πολύπλοκους αυτόματους συμπερασμούς. Συμπερασμούς που βοηθούν σε πληθώρα τομέων όπως η μηχανικά υποβοηθούμενη κατανόηση του προγράμματος, η εύρεση σφαλμάτων, και η βελτιστοποίηση της απόδοσης του προγράμματος.

Τα παραπάνω φανερώνουν ένα από τα βαθύτερα προβλήματα κάθε αλγορίθμου στατικής ανάλυσης δεικτών. Δηλαδή ότι, συχνά, ο σχεδιασμός ενός τέτοιου αλγορίθμου είναι αποτέλεσμα ισορροπίας μεταξύ θεμάτων ακριβείας και κλιμάκωσης. Είναι σχετικά απλό μία ανάλυση να επικεντρωθεί στον υπολογισμό αποτελεσμάτων υψηλής ακρίβειας, θυσιάζοντας την γενική απόδοση του αλγορίθμου. Αντίστοιχα, είναι δυνατόν να σχεδιαστούν πολύ αποδοτικοί αλγόριθμοι, οι \mbox{οποίοι} όμως θα υπολογίζουν μεγάλες εκτιμήσεις των (πραγματικών) αποτελεσμάτων οδηγώντας σε τεράστια ανακρίβεια.

Η διατριβή αυτή στοχεύει στην αντιμετώπιση του παραπάνω προβλήματος, με την κύρια θέση της να συνοψίζεται ώς εξής:

\begin{displayquote}
Είναι δυνατόν να σχεδιαστούν αλγόριθμοι στατικής ανάλυσης δεικτών που παρουσιάζουν \emph{υψηλή ακρίβεια} αλλά και \emph{κλιμάκωση}, εφαρμόζοντας προσεκτικά διαφορετικές πολιτικές σε διαφορετικά σημεία του προγράμματος. Συμπληρωματικά, είναι δυνατόν να σχεδιαστούν αναλύσεις που προσφέρουν \emph{ισχυρές εγγυήσεις εγκυρότητας} των αποτελεσμάτων, αλλά για στοχευμένα κομμάτια του προγράμματος.
\end{displayquote}

Στη συνέχεια, θα παρουσιάσουμε διάφορες τεχνικές για την υλοποίηση αποδοτικών αλγορίθμων ανάλυσης δεικτών, στο περιβάλλον της γλώσσας προγραμματισμού {\en Java}, προσαρμόζοντας προσεκτικά την στρατηγική του κάθε αλγορίθμου σε διαφορετικά σημεία του προγράμματος. Επιπροσθέτως, θα παρουσιάσουμε δύο αλγορίθμους αμυντικής φύσης που στοχεύουν στον \mbox{υπολογισμό} αποτελεσμάτων υψηλής εμπιστοσύνης, ακόμη και εν μέσω ``εχθρικού΄΄ ή άγνωστου κώδικα.


\section*{Βασικές Έννοιες της Στατικής Ανάλυσης Δεικτών}

Πριν κάνουμε μια συνοπτική αναφορά των επιστημονικών συνεισφορών της διατριβής αυτής, είναι απαραίτητο να γίνει μια μικρή εισαγωγή σε βασικές έννοιες της στατικής ανάλυσης (δεικτών), που αποτελούν το επιστημονικό και τεχνικό υπόβαθρο της δουλειάς μας.

\paragraphhead{Πλατφόρμα Υλοποίησης \& Γλώσσα υπό Ανάλυση.}
Το μεγαλύτερο μέρος της δουλειάς που θα παρουσιάσουμε στη συνέχεια, είναι υλοποιημένο στην πλατφόρμα {\en \doop{}}\cite{oopsla:2009:Bravenboer}, χρησιμοποιώντας τη δηλωτική γλώσσα προγραμματισμού {\en Datalog}. Το {\en \doop{}} αποτελεί εδώ και χρόνια μία καλά εδραιωμένη πλατφόρμα ανάπτυξης αλγορίθμων στατικής ανάλυσης δεικτών, προσφέροντας μία πληθώρα αναλύσεων που στοχεύουν προγράμματα {\en Java}. Περισσότερες \mbox{λεπτομέρειες} δίνονται στο Κεφάλαιο~\ref{sec:back:doop}.

Αξίζει να σημειωθεί ότι, αν και οι ιδέες και οι αλγόριθμοι που εξερευνώνται παρακάτω επικεντρώνονται γύρω από την ανάλυση προγραμμάτων {\en Java}, είναι αρκετά πιθανή μία γενίκευση τους, σε μικρότερο ή μεγαλύτερο βαθμό, και σε άλλες γλώσσες προγραμματισμού που προσφέρουν παρόμοια χαρακτηριστικά και ακολουθούν παρόμοια μοντέλα/παραδείγματα.


\paragraphhead{Χρήση Συμφραζομένων.}
Όπως προαναφέρθηκε, η υλοποίηση κάθε πολύπλοκου αλγορίθμου ανάλυσης δεικτών σύντομα καταλήγει σε μία προσπάθεια εξισορρόπησης μεταξύ ακρίβειας και απόδοσης. Στο παρελθόν, η επιστημονική κοινότητα έχει επεκτείνει το οπλοστάσιο της με διάφορες έννοιες και τεχνικές προς διαχείριση αυτής της κατάστασης. Μία τέτοια τεχνική, που στοχεύει στην καταπολέμηση της ανακρίβειας των αποτελεσμάτων, ελπίζοντας χωρίς ταυτόχρονα επιβάρυνση της απόδοσης, είναι αυτή των \emph{συμφραζομένων} ({\en context}). Η χρήση των συμφραζομένων (πρακτικά επιπλέον πληροφορίας) γίνεται επαυξάνοντας στοιχεία της εκάστοτε ανάλυσης (π.χ., τοπικές μεταβλητές, πεδία και μεθόδους) ώστε η ανάλυση να καταφέρει να τα χειριστεί με μεγαλύτερη ακρίβεια.

Η κεντρική ιδέα είναι ότι η ανάλυση θα διαφοροποιήσει τον χειρισμό στοιχείων του προγράμματος όταν αυτά συνδυάζονται με κάποια συμφραζόμενα, ενώ θα τα χειριστεί ομοιογενώς όταν συνδυάζονται με κάποια άλλα. Για παράδειγμα, μία ανάλυση μπορεί να εξετάσει διαφορετικά κάποια μεθόδου όταν η κλήση της έγινε μέσα στη μέθοδο \code{A}, μέσα στη μέθοδο \code{B} ή οπουδήποτε αλλού (δηλαδή παρουσιάζοντας τρεις διαφορετικές τιμές συμφραζομένων).

Δύο βασικές κατηγορίες συμφραζομένων έχουν χρησιμοποιηθεί ευρέως στο παρελθόν: τα συμφραζόμενα \emph{σημείων-κλήσης} (που οδηγούν στις λεγόμενες {\en \emph{call-site sensitive}} αναλύσεις) όπου τα συμφραζόμενα δομούνται από εντολές κλήσης μέσα στον κώδικα του προγράμματος, και τα συμφραζόμενα \emph{αντικειμένων} (που οδηγούν στις λεγόμενες {\en \emph{object sensitive}} αναλύσεις) όπου τα συμφραζόμενα δομούνται από τα αφηρημένα αντικείμενα-παραλήπτες πάνω στα οποία εφαρμόζονται οι τυχόν κλήσεις συναρτήσεων. Περισσότερες λεπτομέρειες δίνονται στο Κεφάλαιο~\ref{sec:back:context}.


\paragraphhead{``{\en May}'' έναντι ``{\en Must}'' Αναλύσεων.}
Όπως αναφέραμε στην αρχή, ο στόχος κάθε αλγορίθμου στατικής ανάλυσης είναι ο αυτόματος συμπερασμός για κάποιο σύνολο συμπεριφορών που δύναται να επιδείξει ένα πρόγραμμα \emph{σε όλες} τις πιθανές εκτελέσεις του. Μια τέτοια προσπάθεια είναι ένα μη-αποφασίσιμο πρόβλημα για τα περισσότερα σύνολα συμπεριφορών παρά για τα πιο τετριμμένα από αυτά (για ένα τυχαίο πρόγραμμα προς ανάλυση). Κατά συνέπεια, κάθε πρακτικός αλγόριθμος αναγκάζεται να κάνει κάποια εκτίμηση του συνόλου των συμπεριφορών προς μία από τις δύο εξής κατευθύνσεις: είτε θα υπολογίσει μία \emph{υπέρ}-εκτίμηση των αποτελεσμάτων και θα αναφέρει όλες τις πιθανές συμπεριφορές του προγράμματος καθώς και κάποιες που δεν είναι δυνατόν να προκύψουν ποτέ, είτε θα υπολογίσει μία \emph{υπό}-εκτίμηση και θα αναφέρει μόνο ένα υποσύνολο των πιθανών συμπεριφορών. Μία αδρή κατηγοριοποίηση των αναλύσεων μπορεί να γίνει κάτω από αυτό το πρίσμα σε {\en may}-αναλύσεις (που αναφέρουν υπερεκτιμήσεις της πραγματικότητας) και σε {\en must}-αναλύσεις (που αναφέρουν υποεκτιμήσεις της πραγματικότητας). Περισσότερα στο Κεφάλαιο~\ref{sec:back:may-must}.


\paragraphhead{Εγκυρότητα Αποτελεσμάτων.}
Ένας θεωρητικός όρος που συχνά χρησιμοποιείται για να χαρακτηρίσει έναν αλγόριθμο στατικής ανάλυσης είναι αυτός της \emph{εγκυρότητας}. Με απλά λόγια, λέμε ότι ένας αλγόριθμος είναι έγκυρος όταν τα αποτελέσματα που υπολογίζει συνάδουν με τους αρχικούς ισχυρισμούς του. Για παράδειγμα, μία {\en may}-ανάλυση δεικτών ισχυρίζεται ότι σκοπεύει να υπολογίσει μία υπερεκτίμηση του συνόλου των αντικειμένων στα οποία μπορεί να δείξει κάθε έκφραση ενός προγράμματος, σε κάθε πιθανή εκτέλεση του. Αν δεν λείπει κάποιο ζευγάρι ``έκφραση/αντικείμενο'', που θα μπορούσε πραγματικά να συμβεί στο πρόγραμμα, από τα αποτελέσματα της ανάλυσης, τότε ο αλγόριθμος χαρακτηρίζεται από εγκυρότητα. Έτσι, μία τετριμμένα έγκυρη, αλλά και παντελώς άχρηστη, {\en may}-ανάλυση δεικτών είναι μία που υπολογίζει το καρτεσιανό γινόμενο κάθε έκφρασης του προγράμματος με κάθε αφηρημένο αντικείμενο.

Ποικίλοι παράγοντες οδηγούν τους περισσότερους αλγορίθμους {\en may}-ανάλυσης δεικτών στο να θυσιάζουν την εγκυρότητα σε κάποιο βαθμό ώστε να καταφέρουν να διατηρήσουν κάποιο ποσοστό κλιμάκωσης. Μια πιο αναλυτική συζήτηση γύρω από το θέμα της εγκυρότητας ακολουθεί στα Κεφάλαια~\ref{sec:back:soundness}-\ref{sec:back:soundiness}.



\section*{Δομή Διατριβής και Επιστημονικές Συνεισφορές}

Το περιεχόμενο της διατριβής δομείται σε εννέα κεφάλαια. Το πρώτο κεφάλαιο δίνει μία σύντομη αναφορά του γενικού χώρου της στατικής ανάλυσης δεικτών, εδραιώνει την κεντρική θέση της διατριβής καθώς και τις επιστημονικές συνεισφορές της, και τέλος, παρουσιάζει την δόμηση που θα ακολουθηθεί στη συνέχεια του κειμένου.

Το δεύτερο κεφάλαιο περιέχει μία σύντομη περιγραφή χρήσιμων και απαραίτητων εννοιών, τεχνικών και εργαλείων από την υπάρχουσα επιστημονική βιβλιογραφία, που αποτελούν την υποκείμενη βάση για την δουλειά μας.


\paragraphhead{Υβριδικές Αναλύσεις Συμφραζομένων.}
Τα συμφραζόμενα αντικειμένων εισήχθησαν το 2002 από την {\en Milanova}~\cite{issta:2002:Milanova} ως εναλλακτική των συμφραζομένων σημείων-κλήσης. Από τότε υπάρχει πληθώρα ενδείξεων ότι αποτελούν τη βέλτιστη επιλογή είδους συμφραζομένων, όσον αφορά προγράμματα εκφρασμένα σε αντικειμενοστρεφείς γλώσσες, εξασφαλίζοντας υψηλή ακρίβεια με χαμηλότερο συγκριτικά κόστος. Τόσο μεγάλη ήταν η επιτυχία τους που έχουν πρακτικά αντικαταστήσει την κλασική εναλλακτική των σημείων-κλήσης. Παρ' όλα αυτά, τα συμφραζόμενα σημείων-κλήσης δεν είναι πάντα υποδεέστερα καθώς υπάρχουν συγκεκριμένα χαρακτηριστικά γλωσσών και μοτίβα προγραμματισμού που ευνοούν αυτή την επιλογή.

Συνεπώς, δεν είναι παράλογη μία προσέγγιση όπου και τα δύο είδη συμφραζομένων συνδυάζονται, ομοιογενώς, με κάπως αφελή τρόπο, σε κάθε σημείο του προγράμματος στοχεύοντας ώστε τα οφέλη στην ακρίβεια να είναι ακόμα μεγαλύτερα. Όντως, ένας τέτοιος συνδυασμός έχει σαν αποτέλεσμα κάποια βελτίωση στον τομέα της ακρίβειας, αλλά στις περισσότερες των περιπτώσεων μία τέτοια βελτίωση συνοδεύεται με ένα απαγορευτικά υψηλό κόστος.

Απόρροια αυτής της παρατήρησης είναι η πρώτη μας επιστημονική συνεισφορά, που παρουσιάζεται στο τρίτο κεφάλαιο. Εκεί περιγράφουμε μία προσπάθεια προς έναν πιο εκλεπτυσμένο συνδυασμό των δύο ειδών συμφραζομένων. Η \emph{υβριδική} μας προσέγγιση οδηγεί σε μία οικογένεια αναλύσεων όπου τα διαφορετικά είδη συμφραζομένων συνδυάζονται μόνο σε συγκεκριμένα σημεία του προγράμματος, ώστε η ακρίβεια της ανάλυσης να έχει τα οφέλη της ύπαρξης όλων των ειδών χωρίς όμως να χρειάζεται να πληρώσει και το αντίστοιχο κόστος.

Πιο συγκεκριμένα, η κεντρική ιδέα των υβριδικών αλγορίθμων μας έγκειται στη χρήση των συμφραζομένων αντικειμένων σαν το κυρίαρχο είδος, για την ανάλυση αντικειμενοστρεφών χαρακτηριστικών του προγράμματος στα οποία και προσφέρουν τα περισσότερα οφέλη, και τον συνδυασμό τους με την πιο κλασική εναλλακτική των συμφραζομένων σημείων-κλήσης εκεί που τα πρώτα υστερούν. Το πιο χαρακτηριστικό τέτοιο σημείο προγράμματος είναι η κλήση στατικών συναρτήσεων, όπου και δεν υπάρχει η κατάλληλη πληροφορία που χρειάζονται τα συμφραζόμενα αντικειμένων. Οι κλασικοί αλγόριθμοι ανάλυσης δεικτών αντιμετωπίζουν το πρόβλημα αυτό μεταφέροντας πληροφορία από το πιο πρόσφατο κατάλληλο σημείο (που μπορεί να απέχει από το σημείο της στατικής κλήσης), με την ελπίδα ότι η πληροφορία αυτή θα αποβεί χρήσιμη ξανά στο μέλλον. Στο ενδιάμεσο όμως, η επιπλέον αυτή πληροφορία προσφέρει ελάχιστα στη γενική ακρίβεια του αλγορίθμου ενώ η ύπαρξη της δεν έρχεται χωρίς (κάποιες φορές βαρύ) κόστος. Μία υβριδική αντιμετώπιση θα επιλέξει για τα σημεία αυτά (και μόνο) την χρήση συμφραζομένων σημείων-κλήση, τα οποία είναι ικανά να βελτιώσουν την τοπική ακρίβεια της ανάλυσης χωρίς να την επιβαρύνουν σημαντικά.

Σαν αποτέλεσμα, αυτός ο επιλεκτικός συνδυασμός συμφραζομένων οδηγεί σε αναλύσεις σημαντικά ανώτερες όχι μόνο συγκριτικά με αυτές που ακολουθούν κάποιο αφελή συνδυασμό συμφραζομένων, αλλά και ακόμα σε σχέση με τις κλασικές, ``κανονικές'', μη-υβριδικές αναλύσεις. Αυτό προκύπτει συμπερασματικά με τη συλλογή εκτεταμένων πειραματικών δεδομένων από μεγάλα προγράμματα {\en Java}. Για παράδειγμα, σε σύγκριση με μία αρκετά διαδεδομένη και χρήσιμη ανάλυση που χρησιμοποιεί συμφραζόμενα αντικειμένων μήκους δύο, η προσέγγιση μας προσφέρει επιταχύνσεις της τάξης του {\en \nums{1.53x}} αλλά και καλύτερη ακρίβεια.


\paragraphhead{Ανάλυση Ενδοσκόπησης.}
Το τέταρτο κεφάλαιο παρουσιάζει τη δεύτερη επιστημονική συνεισφορά μας, γύρω από την προσπάθεια των αλγορίθμων ανάλυσης δεικτών να επιτύχουν καλή απόδοση και κλιμάκωση χωρίς να εγκαταλείψουν την ακρίβεια των αποτελεσμάτων. \mbox{Όμως}, είναι συχνό φαινόμενο στο χώρο αυτό, οι αναλύσεις να βρίσκονται σε ένα εκ των δύο άκρων του φάσματος: είτε έχουν αρκετή ακρίβεια ώστε το σύνολο των δεδομένων υπό ανάλυση να παραμένει διαχειρίσιμο και σαν αποτέλεσμα να επιτυγχάνουν μία εντυπωσιακή κλιμάκωση, είτε γρήγορα εκτροχιάζονται στο πρώτο σημάδι σημαντικής ανακρίβειας και καταλήγουν να \mbox{είναι} τάξεις μεγέθους πιο κοστοβόρες σε σχέση με το αναμενόμενο με βάση το μέγεθος του αναλυόμενου προγράμματος.

Προς την αντιμετώπιση αυτού του ζητήματος κινείται και η προσέγγιση μας σε αυτό το κεφάλαιο, προτείνοντας μία ανάλυση \emph{ενδοσκόπησης} που απαρτίζεται από δύο βήματα. Η προσέγγιση αυτή επιτρέπει στην ανάλυση να παρουσιάζει μία ομοιόμορφη κλιμάκωση σε μεγάλα προγράμματα {\en Java}, εξαλείφοντας τα προβληματικά, πιθανά φαινόμενα απόδοσης, έχοντας μόνο ένα μικρό αρνητικό αντίκτυπο στην τελική ακρίβεια.

Η ανάλυσή μας εφαρμόζει ένα γνωστό μοτίβο: πρώτα εκτελεί μία ανάλυση που δεν χρησιμοποιεί συμφραζόμενα, και άρα είναι ανακριβής αλλά και φτηνή και γρήγορη, και στη συνέχεια, με βάση τα δεδομένα που συλλέχθηκαν στην πρώτη φάση, εκτελεί μία πιο εκλεπτυσμένη ανάλυση (δηλαδή, με χρήση κάποιου είδους συμφραζομένων) αλλά μόνο για συγκεκριμένα σημεία του προγράμματος. Η κατεύθυνση αυτή ευελπιστεί ότι η επιπλέον ακρίβεια στα σημεία αυτά δεν θα είναι τελικά απαγορευτική για την συνολική απόδοση του αλγορίθμου. Άρα, η πρόκληση του όλου εγχειρήματος βρίσκεται στην κατάλληλη επιλογή αυτών των σημείων. Δείχνουμε ότι μία πειθαρχημένη προσέγγιση μπορεί να προβεί αρκετά αποτελεσματική, επιφέροντας κλιμάκωση με σημαντικές βελτιώσεις στην απόδοση σε προγράμματα που στο παρελθόν δεν ήταν δυνατόν να αναλυθούν με ακρίβεια.

Στο κεφάλαιο αυτό, παρουσιάζουμε διάφορες μετρικές για την αξιολόγηση της πληροφορίας από την πρώτη φάση, και στη συνέχεια δύο ευριστικές για την μετέπειτα επιλογή των κατάλληλων σημείων του προγράμματος που θα αναλυθούν με μεγαλύτερη ακρίβεια στη δεύτερη φάση.

Συλλέγοντας αρκετά πειραματικά δεδομένα, επιβεβαιώνουμε τα οφέλη μίας ανάλυσης ενδοσκόπησης. Εξερευνούμε τις πιθανές απώλειες σε ακρίβεια αλλά και τις βελτιώσεις σε απόδοση και κλιμάκωση, δοκιμάζοντας διαφορετικές παραμέτρους στις μετρικές και ευριστικές μας. \mbox{Τελικά}, εξακριβώνουμε ότι, ακόμα και με παραμέτρους που στοχεύουν σε υψηλή ακρίβεια, η ανάλυση μας είναι αποτελεσματική στην διαχείριση προβληματικών περιπτώσεων, που προηγουμένως ήταν αδύνατον να αναλυθούν χωρίς τρομακτική απώλεια ακρίβειας. Αυτά τα πειραματικά συμπεράσματα εδραιώνουν την εμπιστοσύνη μας στον ισχυρισμό ότι οι αναλύσεις συμφραζομένων μπορούν να χρησιμοποιηθούν ευρέως και όχι απλά μεμονωμένα σε εκείνες τις περιπτώσεις που ``δουλεύουν αρκετά καλά''.


\vspace{5 mm}
Στη συνέχεια της διατριβής, στρέφουμε την προσοχή μας σε αναλύσεις που επικεντρώνονται στον υπολογισμό αποτελεσμάτων τα οποία συνοδεύονται με μεγάλη εμπιστοσύνη. Αν και αυτό οδηγεί σε συντηρητικές, αμυντικές αναλύσεις, συχνά απρόθυμες να προβούν σε νέους συμπερασμούς, όταν τελικά αναφέρουν κάποιο αποτέλεσμα το κάνουν με ισχυρή βεβαιότητα στην εγκυρότητα του.


\paragraphhead{{\en Must}-Ανάλυση Συνωνύμων - Ένα Λογικό Μοντέλο.}
Το πέμπτο κεφάλαιο, στο οποίο περιγράφεται η τρίτη επιστημονική συνεισφορά μας, αρχίζει μία εξερεύνηση προς μία διαφορετική κατεύθυνση. Πρώτον, αντί για μία ανάλυση δεικτών, παρουσιάζουμε μία ανάλυση \emph{συνωνύμων} ({\en aliases}). Μία ανάλυση αυτού του είδους έχει σκοπό τον υπολογισμό των εκφράσεων ενός προγράμματος που αποτελούν συνώνυμα, δηλαδή δείχνουν στο ίδιο αντικείμενο στη μνήμη. Οι αναλύσεις συνωνύμων συνδέονται στενά με τις αναλύσεις δεικτών, αλλά έχουν και βασικές διαφορές. Δεύτερον, η ανάλυση που προτείνουμε ανήκει στην οικογένεια των {\en must-}αναλύσεων, υπολογίζει δηλαδή μία υποεκτίμηση της πραγματικότητας. Αυτό συνεπάγεται ότι, μία τέτοια ανάλυση αποτυγχάνει να υπολογίσει κάποια ισχύοντα ζευγάρια συνωνύμων, αλλά αυτά τα οποία τελικά θα υπολογίσει είναι σίγουρο ότι ισχύουν.

Οι ισχυρές βεβαιώσεις που συνοδεύουν μία ανάλυση αυτού του είδους, καθιστούν τα αποτελέσματα της ιδανικά για αρκετές εφαρμογές:
\begin{inparaenum}[(1)]
\item είναι σημαντικά για πληθώρα βελτιστοποιήσεων σε μεταγλωττιστές,
\item μπορούν να βελτιώσουν την ακρίβεια προγραμμάτων για τον έλεγχο σφαλμάτων, για παράδειγμα, ανιχνευτές μη-τερματισμού ή λάθους δεικτοδότησης ({\en null-reference}) σε εκφράσεις του προγράμματος,
\item μπορούν να χρησιμοποιηθούν σαν δομικά στοιχεία πιο σύνθετων και πολύπλοκων αναλύσεων, και
\item μπορούν να προβούν ανεκτίμητα στην άμεση κατανόηση του προγράμματος από τον προγραμματιστή.
\end{inparaenum}

Για να καταφέρει μία ανάλυση να επιδείξει τόσο υψηλή εμπιστοσύνη στα αποτελέσματα της, χρειάζεται να σέβεται την ροή του προγράμματος, να διατηρεί σύνολα αποτελεσμάτων ξεχωριστά για κάθε σημείο του προγράμματος, και να μεταφέρει σε επόμενα σημεία μόνο όποιο υποσύνολο της πληροφορίας συνεχίζει να ισχύει. Με μία πρώτη ματιά, μία τέτοια προσέγγιση φαντάζει αυτονόητη, αλλά οι περισσότερες ({\en may}-) αναλύσεις δεικτών δεν κάνουν αυτή την επιλογή καθώς κάτι τέτοιο προσφέρει λίγο στην ακρίβεια τους και ταυτόχρονα επιβαρύνει αρκετά την απόδοση τους. Στην περίπτωση μίας αμυντικής ανάλυσης όμως, αυτή η κατεύθυνση είναι παραπάνω από απαραίτητη.

Έτσι, αρχικά, στο πέμπτο κεφάλαιο παρουσιάζουμε ένα μινιμαλιστικό μοντέλο της ανάλυσης μας, εκφρασμένο στη δηλωτική γλώσσα {\en Datalog}. Το μοντέλο είναι αρκετά εκλεπτυσμένο για να περιγράψει τα κύρια χαρακτηριστικά που πρέπει να διαθέτει μία {\en must-}ανάλυση συνωνύμων, αλλά ταυτόχρονα και αρκετά απλό ώστε να μπορούμε να επικεντρωθούμε στην ουσία της ανάλυσης και όχι στους πολύπλοκους τρόπους με τους οποίους αλληλεπιδρούν τα διάφορα στοιχεία της γλώσσας που αναλύουμε.

Επιπροσθέτως, αξίζει να σημειωθεί ότι το μοντέλο μας παρουσιάζει μία μη συμβατική χρήση των συμφραζομένων. Οι κλασικές αναλύσεις εφαρμόζουν τα συμφραζόμενα σε μία προσπάθεια για βελτίωση της ακρίβειας, δηλαδή σαν ένα επιπλέον, θετικό αλλά προαιρετικό στοιχείο. Η ανάλυση μας χρησιμοποιεί τα συμφραζόμενα σαν ένα εργαλείο εξασφάλισης της εγκυρότητας των αποτελεσμάτων. Όποτε χρειάζεται να εξερευνήσει πέρα από τα τοπικά όρια μίας συνάρτησης, το κάνει επεκτείνοντας τα υπάρχοντα συμφραζόμενα, στο μέτρο που κάτι τέτοιο επιτρέπεται από τις παραμέτρους της ανάλυσης. Όταν κάτι τέτοιο δεν είναι πια δυνατόν, σταματάει κάθε προσπάθεια για συμπερασμούς καθώς αυτοί θα οδηγούσαν σε μη έγκυρα αποτελέσματα. Συνεπώς, στην προσέγγιση μας τα συμφραζόμενα αποτελούν αναπόσπαστο κομμάτι του μοντέλου, επιτρέποντας τους συμπερασμούς της ανάλυσης να υπερβούν τα στενά όρια κάθε συνάρτησης.

Ένα ακόμη θετικό στοιχείο της μοντελοποίησης μας είναι το γεγονός ότι η απουσία κομματιών του υπό-ανάλυση προγράμματος (π.χ., κώδικα βιβλιοθηκών) δεν έχει κάποιο αρνητικό αντίκτυπο στην συνολική εγκυρότητα της ανάλυσης. Η παρουσία πιθανώς επιπλέον κώδικα οδηγεί στο συμπερασμό ακόμα περισσότερων αποτελεσμάτων, χωρίς όμως κάτι τέτοιο να ακυρώνει τους προηγούμενους υπολογισμούς της ανάλυσης.


\paragraphhead{{\en Must}-Ανάλυση Συνωνύμων - Ειδικές Δομές Δεδομένων.}
Η προσεκτική παρατήρηση του παραπάνω μοντέλου αποκαλύπτει διάφορα σημαντικά χαρακτηριστικά μίας {\en must-}ανάλυσης συνωνύμων, αλλά και τις ανάγκες που πρέπει να καλύψει κάθε πιθανή υλοποίηση. Σαν αποτέλεσμα, στο έκτο κεφάλαιο παρουσιάζουμε μία ειδική δομή δεδομένων, η οποία αξιοποιώντας αυτές τις παρατηρήσεις επιφέρει σημαντικές βελτιώσεις στην απόδοση και την κλιμάκωση της ανάλυσης.

Πιο συγκεκριμένα, η πρώτη παρατήρηση είναι ότι όταν μία {\en must-}ανάλυση υπολογίζει την πληροφορία συνωνύμων, στην πράξη διατηρεί μία σχέση ισοδυναμίας.\footnote{Αντιθέτων, όπως περιγράφουμε πιο αναλυτικά στο κείμενο του κεφαλαίου, η σχέση συνωνύμων σε μία {\en may-}ανάλυση \emph{δεν} αποτελεί σχέση ισοδυναμίας.} Για παράδειγμα, αν η ανάλυση υπολογίσει ότι οι μεταβλητές {\en \code{x}} και {\en \code{y}} αποτελούν συνώνυμα, και το ίδιο αντίστοιχα και οι μεταβλητές {\en \code{y}} και {\en \code{z}}, τότε πρέπει αυτομάτως να συμπεριλάβει και το ζευγάρι \{{\en \code{x}} με {\en \code{z}}\} (καθώς και όλα τα συμμετρικά, \{{\en \code{y}} με {\en \code{x}}\}, \{{\en \code{z}} με {\en \code{y}}\}, και \{{\en \code{z}} με {\en \code{x}}\}). Μία ρητή αναπαράσταση όλων των ζευγαριών μπορεί να επιφέρει σημαντική επιβάρυνση στην απόδοση του αλγορίθμου, ειδικά όταν αυτό συνδυαστεί και με την επόμενη παρατήρηση.

Η δεύτερη παρατήρηση έχει να κάνει με το γεγονός ότι η ανάλυση μας δεν περιορίζεται σε συμπερασμούς για εκφράσεις του προγράμματος μήκους ένα (δηλαδή, τοπικές μεταβλητές), αλλά συμπεριλαμβάνει και μεγαλύτερες, όπως για παράδειγμα η {\en \code{obj.fld1.fld2}} (μέχρι κάποιο μέγιστο μήκος, παράμετρο της ανάλυσης). Το υποκείμενο πρόβλημα είναι το εξής: για παράδειγμα, αν η ανάλυση έχει υπολογίσει ότι δύο μεταβλητές {\en \code{x}} και {\en \code{y}} είναι συνώνυμα, τότε πρέπει επίσης να υπολογίσει ρητά και ζευγάρια σαν τα \{{\en \code{x.f}} με {\en \code{y.f}}\}, \{{\en \code{x.g}} με {\en \code{y.g}}\}, \{{\en \code{x.f.h}} με {\en \code{y.f.h}}\}, κτλ. Ένας τέτοιος συμπερασμός οδηγεί σε εκθετικό πλήθος ζευγαριών.

Για τους παραπάνω λόγους, εισάγουμε μία ειδική δομή δεδομένων, η οποία αποτυπώνει τη σχέση συνωνύμων που υπολογίζει μία {\en must}-ανάλυση, και ταυτόχρονα αξιοποιεί τις παραπάνω παρατηρήσεις με αποτέλεσμα τη σημαντική βελτίωση της συνολικής απόδοσης. Η δομή μας έχει τη μορφή κατευθυνόμενου γράφου, όπου κάθε κόμβος αναπαριστά ομάδες μεταβλητών (δηλαδή, τάξεις ισοδυναμίας), με κάθε μεταβλητή-μέλος να είναι συνώνυμη με όλες τις άλλες που βρίσκονται στον ίδιο κόμβο. Κάθε ακμή αναπαριστά την πρόσβαση σε κάποιο πεδίο, ενώ η φορά της ακμής κωδικοποιεί ποιός κόμβος ``δείχνει'' σε ποιόν (με τον ίδιο τρόπο που συναντάται η έννοια σε μία ανάλυση δεικτών). Σύνθετες εκφράσεις του προγράμματος, μήκους μεγαλύτερου του ένα, κωδικοποιούνται έμμεσα στα μονοπάτια που εμφανίζονται μεταξύ των κόμβων του γράφου.

Τέλος, επιβεβαιώνουμε πειραματικά τα θεωρητικά οφέλη της ειδικής δομής που παρουσιάσαμε, παρατηρώντας βελτιώσεις στην απόδοση του αλγορίθμου κατά δύο τάξεις μεγέθους. Με αυτή την προσέγγιση, καθίσταται δυνατή η αποδοτική εφαρμογή της {\en must}-ανάλυσης συνωνύμων σε μεγάλα προγράμματα {\en Java}, με χρόνους εκτέλεσης συχνά κάτω από μισό λεπτό.


\paragraphhead{Αμυντική Ανάλυση Δεικτών.}
Στο έβδομο κεφάλαιο ολοκληρώνουμε την παρουσίαση των επιστημονικών συνεισφορών της διατριβής, περιγράφοντας την τέταρτη και τελευταία \mbox{από} αυτές. Επιστρέφουμε ξανά στην οικογένεια των {\en may}-αναλύσεων δεικτών, αυτή τη φορά με κυρίαρχο στόχο τον υπολογισμό πραγματικά έγκυρων αποτελεσμάτων, ακόμα και υπό την παρουσία άγνωστου ή ``εχθρικού'' κώδικα. Η ανάλυση μας δεν θέτει κάποιο περιορισμό στα \mbox{χαρακτηριστικά} που χρησιμοποιεί το υπό-ανάλυση πρόγραμμα, και επίσης προσπαθεί να μην κάνει εκπτώσεις στην συνολική απόδοση του αλγορίθμου για να επιτύχει τους στόχους της.

Η ανάλυση μας, όντας μέλος της οικογένειας των {\en may}-αναλύσεων, έχει σαν στόχο τον υπολογισμό μίας υπερεκτίμησης της πραγματικής συμπεριφοράς του προγράμματος. Για να καταφέρει ταυτόχρονα να τηρήσει τους ισχυρισμούς εγκυρότητας που δίνει, θα πρέπει όταν για οποιοδήποτε λόγο δεν είναι σίγουρη για το σύνολο αντικειμένων στα οποία μπορεί να δείξει κάποια έκφραση, να αναφέρει ότι μπορεί να δείξει στα πάντα.

Μία αφελής προσέγγιση του θέματος οδηγεί στον περιττό υπολογισμό αρκετών αποτελεσμάτων, τα οποία στη συνέχεια θα ακυρωθούν ή έμμεσα θα υποσκελιθούν από άλλα, ώστε τελικά να τηρηθεί η εγκυρότητα. Κάτι τέτοιο έχει φυσικά σημαντικές, αρνητικές επιπτώσεις στην \mbox{συνολική} απόδοση της ανάλυσης. Η προσέγγισή μας καταφέρνει να παραμείνει αποδοτική χωρίς να ζημιώνει την συνολική εγκυρότητα, αναβάλλοντας τον υπολογισμό πληροφορίας μέχρις ότου είναι βέβαιη ότι αυτή δεν θα ακυρωθεί σε μετέπειτα στάδιο.

Σαν αποτέλεσμα, οι παραπάνω σχεδιαστικές επιλογές επιτρέπουν στην ανάλυση μας να είναι αρκετά αποδοτική, επιτυγχάνοντας υψηλά επίπεδα ακριβείας, αδύνατα για τις κλασικές, προϋπάρχουσες αναλύσεις. Παρά την αρκετά συντηρητική και αμυντική φύση της, η ανάλυση καταφέρνει να δώσει έγκυρα και άμεσα εφαρμόσιμα αποτελέσματα για ένα μεγάλο υποσύνολο του υπό-ανάλυση προγράμματος. Πειραματικά, κάτω από τις πιο απαισιόδοξες και αμυντικές παραμέτρους, γίνεται κάλυψη του {\en \nums{34-74\%}} του προγράμματος σε σύγκριση με μία από τις καλύτερες, αλλά μη-έγκυρες, αναλύσεις του χώρου.


\vspace{5 mm}
Τέλος, στα εναπομείναντα κεφάλαια της διατριβής, δίνεται ο επίλογος της εν λόγω δουλειάς. Στο όγδοο κεφάλαιο διερευνούμε σχετική ερευνητική δουλειά του χώρου, για τις τέσσερεις επιστημονικές συνεισφορές μας. Κλείνοντας, στο ένατο και τελευταίο κεφάλαιο σκιαγραφώνται μελλοντικές ερευνητικές κατευθύνσεις και γίνεται μία τελική εκτίμηση της διατριβής.