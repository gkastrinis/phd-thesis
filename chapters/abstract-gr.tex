Η στατική ανάλυση στοχεύει στον αυτόματο συμπερασμό ιδιοτήτων που κάποιο πρόγραμμα μπορεί να επιδείξει σε κάθε πιθανή εκτέλεση, χωρίς στην πράξη να εκτελείται. Η στατική ανάλυση \emph{δεικτών} αποτελεί μια μεγάλη υποκατηγορία της που επικεντρώνεται στα δυναμικά αντικείμενα που δύνανται να ``δείξουν'' οι εκφράσεις ενός προγράμματος σε κάποια εκτέλεση του. Η εξέλιξη των γλωσσών προγραμματισμού με την πάροδο των χρόνων οδήγησε στην προσθήκη πολλών επιπέδων αφαίρεσης, τα οποία σαν αποτέλεσμα έχουν ο αυτόματος συμπερασμός για κάποιο πρόγραμμα να αποτελεί τουλάχιστον μία πρόκληση αν όχι και μία αδύνατη προσπάθεια. Συνεπώς, κάθε πρακτικός αλγόριθμος στατικής ανάλυσης πρέπει να στοχεύσει σε μια εκτίμηση των πραγματικών αποτελεσμάτων με κάποια μορφή ανακρίβειας---είτε υπολογίζοντας περισσότερα είτε λιγότερα.

Σε αυτή τη διατριβή παρουσιάζουμε πώς μπορούμε να σχεδιάσουμε \emph{ακριβείς} και συνάμα \emph{αποδοτικούς} αλγορίθμους ανάλυσης δεικτών εφαρμόζοντας διαφορετικές πολιτικές σε διαφορετικά σημεία του προγράμματος. Συμπληρωματικά, δεδομένου ότι ένας αλγόριθμος ανάλυσης δεικτών με βεβαιώσεις εγκυρότητας για όλα τα σημεία του προγράμματος καθώς και πρακτικά αποτελέσματα δεν αποτελεί ρεαλιστική κατεύθυνση, δείχνουμε πώς μπορούμε να σχεδιάσουμε αναλύσεις με \emph{ισχυρές βεβαιώσεις εγκυρότητας} για συγκεκριμένα κομμάτια ενός προγράμματος.

Προηγούμενοι αλγόριθμοι για ανάλυση δεικτών εισήγαγαν την έννοια των \emph{συμφραζομένων} ({\en context}) για να αντιμετωπίσουν το αυξανόμενο πρόβλημα της ανακρίβειας έναντι της αποδοτικότητας. Τα συμφραζόμενα χρησιμοποιούνται για να επαυξήσουν στοιχεία της ανάλυσης ώστε η ανάλυση να καταφέρει να είναι πιο ακριβής χωρίς ταυτόχρονα να πρέπει να κάνει θυσίες στον τομέα της αποδοτικότητας. Παρουσιάζουμε επωφελείς τρόπους συνδυασμού διάφορων ειδών συμφραζομένων σε διαφορετικά σημεία του προγράμματος, χωρίς αυτοί οι συνδυασμοί να επιφέρουν το κόστος που θα παρουσίαζε μία αφελής προσέγγιση.

Μία δεύτερη απόπειρα για δημιουργία αναλύσεων που παρουσιάζουν υψηλή ακρίβεια και αποδοτικότητα μας οδηγεί σε μια ανάλυση \emph{ενδοσκόπησης} ({\en introspection}). Εφαρμόζουμε ένα σύνηθες μοτίβο στο οποίο μια φτηνή ανακριβής ανάλυση εφαρμόζεται πρώτη ώστε να συλλέξει διάφορες μετρικές για το πρόγραμμα, και στη συνέχεια μια δεύτερη πιο ακριβής (και ακριβή) ανάλυση μπορεί να εφαρμοστεί μόνο σε συγκεκριμένα σημεία του προγράμματος---υπό την υπόθεση ότι η πιο ακριβής μεταχείριση των υπολοίπων θα είχε μόνο αρνητικά αποτελέσματα στην συνολική απόδοση.

Εν συνεχεία, μετατοπίζουμε την προσοχή μας προς μια ανάλυση που \emph{υπό}-εκτιμά τα αποτελέσματα της (σε αντίθεση με το σύνηθες των αναλύσεν που υπολογίζουν μία \emph{υπέρ}-εκτίμηση). Με αυτή την αντιμετώπιση, η ανάλυση μας αναφέρει λιγότερα αποτελέσματα αλλά μπορεί να παρέχει ισχυρές βεβαιώσεις ότι αυτά θα ισχύουν πάντα. Βασιζόμενοι πάνω σε παρατηρήσεις για τις ιδιότητες που παρουσιάζουν αναλύσεις αυτού του είδους, εφαρμόζουμε μια ειδική δομή δεδομένων η οποία επιφέρει επιταχύνσεις στον αλγόριθμο μας σχεδόν κατά δύο τάξεις μεγέθους.

Τέλος, στην τέταρτη συνεισφορά της διατριβής, επιστρέφουμε ξανά στην οικογένεια αναλύσεων που υπερεκτιμούν τα αποτελέσματα τους. Ο στόχος μας είναι η δημιουργία ενός αρκετά αποδοτικού αλγορίθμου που όντως παράγει έγκυρα αποτελέσματα χωρίς περιορισμούς στο υποκείμενο πρόγραμμα. Κατά συνέπεια, αυτό μας οδηγεί σε μία συντηρητική ανάλυση, που μπορεί να παρέχει βεβαιώσεις εγκυρότητας ακόμα και ύπο την παρουσία άγνωστου κώδικα, αλλά ταυτόχρονα αποφεύγει την σπατάλη υπολογισμών σε δεδομένα που αργότερα θα χρειαστεί να ανατραπούν για την διατήρηση των βεβαιώσεων αυτών.